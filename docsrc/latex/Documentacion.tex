%% Generated by Sphinx.
\def\sphinxdocclass{report}
\documentclass[letterpaper,10pt,spanish]{sphinxmanual}
\ifdefined\pdfpxdimen
   \let\sphinxpxdimen\pdfpxdimen\else\newdimen\sphinxpxdimen
\fi \sphinxpxdimen=.75bp\relax
\ifdefined\pdfimageresolution
    \pdfimageresolution= \numexpr \dimexpr1in\relax/\sphinxpxdimen\relax
\fi
%% let collapsible pdf bookmarks panel have high depth per default
\PassOptionsToPackage{bookmarksdepth=5}{hyperref}

\PassOptionsToPackage{booktabs}{sphinx}
\PassOptionsToPackage{colorrows}{sphinx}

\PassOptionsToPackage{warn}{textcomp}
\usepackage[utf8]{inputenc}
\ifdefined\DeclareUnicodeCharacter
% support both utf8 and utf8x syntaxes
  \ifdefined\DeclareUnicodeCharacterAsOptional
    \def\sphinxDUC#1{\DeclareUnicodeCharacter{"#1}}
  \else
    \let\sphinxDUC\DeclareUnicodeCharacter
  \fi
  \sphinxDUC{00A0}{\nobreakspace}
  \sphinxDUC{2500}{\sphinxunichar{2500}}
  \sphinxDUC{2502}{\sphinxunichar{2502}}
  \sphinxDUC{2514}{\sphinxunichar{2514}}
  \sphinxDUC{251C}{\sphinxunichar{251C}}
  \sphinxDUC{2572}{\textbackslash}
\fi
\usepackage{cmap}
\usepackage[T1]{fontenc}
\usepackage{amsmath,amssymb,amstext}
\usepackage{babel}



\usepackage{tgtermes}
\usepackage{tgheros}
\renewcommand{\ttdefault}{txtt}



\usepackage[Sonny]{fncychap}
\ChNameVar{\Large\normalfont\sffamily}
\ChTitleVar{\Large\normalfont\sffamily}
\usepackage{sphinx}

\fvset{fontsize=auto}
\usepackage{geometry}


% Include hyperref last.
\usepackage{hyperref}
% Fix anchor placement for figures with captions.
\usepackage{hypcap}% it must be loaded after hyperref.
% Set up styles of URL: it should be placed after hyperref.
\urlstyle{same}

\addto\captionsspanish{\renewcommand{\contentsname}{Contenidos:}}

\usepackage{sphinxmessages}
\setcounter{tocdepth}{1}



\title{Documentacion de cuaderno de practicas}
\date{08 de enero de 2025}
\release{1.0}
\author{Enrique Gómez}
\newcommand{\sphinxlogo}{\vbox{}}
\renewcommand{\releasename}{Versión}
\makeindex
\begin{document}

\ifdefined\shorthandoff
  \ifnum\catcode`\=\string=\active\shorthandoff{=}\fi
  \ifnum\catcode`\"=\active\shorthandoff{"}\fi
\fi

\pagestyle{empty}
\sphinxmaketitle
\pagestyle{plain}
\sphinxtableofcontents
\pagestyle{normal}
\phantomsection\label{\detokenize{index::doc}}


\sphinxstepscope

\sphinxAtStartPar
Documentacion de las practicas
=

\sphinxstepscope


\chapter{Practica 3}
\label{\detokenize{pr3:practica-3}}\label{\detokenize{pr3::doc}}\label{\detokenize{pr3::doc}}

\section{Practica 3.1.1}
\label{\detokenize{pr3:module-pr3.1_1}}\label{\detokenize{pr3:practica-3-1-1}}\index{module@\spxentry{module}!pr3.1\_1@\spxentry{pr3.1\_1}}\index{pr3.1\_1@\spxentry{pr3.1\_1}!module@\spxentry{module}}
\sphinxAtStartPar
Calcular el area de un circulo de radio r
\index{area\_circulo() (en el módulo pr3.1\_1)@\spxentry{area\_circulo()}\spxextra{en el módulo pr3.1\_1}}

\begin{fulllineitems}
\phantomsection\label{\detokenize{pr3:pr3.1_1.area_circulo}}
\pysigstartsignatures
\pysiglinewithargsret{\sphinxcode{\sphinxupquote{pr3.1\_1.}}\sphinxbfcode{\sphinxupquote{area\_circulo}}}{\sphinxparam{\DUrole{n}{r}}}{}
\pysigstopsignatures
\sphinxAtStartPar
Calcula el área de un círculo dado su radio.
\begin{quote}\begin{description}
\sphinxlineitem{Parámetros}
\sphinxAtStartPar
\sphinxstyleliteralstrong{\sphinxupquote{r}} (\sphinxstyleliteralemphasis{\sphinxupquote{float}}) \textendash{} El radio del círculo.

\sphinxlineitem{Devuelve}
\sphinxAtStartPar
El área del círculo calculada como π * r\textasciicircum{}2.

\sphinxlineitem{Tipo del valor devuelto}
\sphinxAtStartPar
float

\end{description}\end{quote}

\end{fulllineitems}

\index{main() (en el módulo pr3.1\_1)@\spxentry{main()}\spxextra{en el módulo pr3.1\_1}}

\begin{fulllineitems}
\phantomsection\label{\detokenize{pr3:pr3.1_1.main}}
\pysigstartsignatures
\pysiglinewithargsret{\sphinxcode{\sphinxupquote{pr3.1\_1.}}\sphinxbfcode{\sphinxupquote{main}}}{}{}
\pysigstopsignatures
\sphinxAtStartPar
Solicita al usuario el radio de un círculo, calcula el área
y muestra el resultado en pantalla.

\end{fulllineitems}



\section{Practica 3.1.2}
\label{\detokenize{pr3:module-pr3.1_2}}\label{\detokenize{pr3:practica-3-1-2}}\index{module@\spxentry{module}!pr3.1\_2@\spxentry{pr3.1\_2}}\index{pr3.1\_2@\spxentry{pr3.1\_2}!module@\spxentry{module}}\index{main() (en el módulo pr3.1\_2)@\spxentry{main()}\spxextra{en el módulo pr3.1\_2}}

\begin{fulllineitems}
\phantomsection\label{\detokenize{pr3:pr3.1_2.main}}
\pysigstartsignatures
\pysiglinewithargsret{\sphinxcode{\sphinxupquote{pr3.1\_2.}}\sphinxbfcode{\sphinxupquote{main}}}{}{}
\pysigstopsignatures
\sphinxAtStartPar
Solicita al usuario el radio de una esfera, calcula su volumen
y muestra el resultado en pantalla.

\end{fulllineitems}

\index{volumen\_esfera() (en el módulo pr3.1\_2)@\spxentry{volumen\_esfera()}\spxextra{en el módulo pr3.1\_2}}

\begin{fulllineitems}
\phantomsection\label{\detokenize{pr3:pr3.1_2.volumen_esfera}}
\pysigstartsignatures
\pysiglinewithargsret{\sphinxcode{\sphinxupquote{pr3.1\_2.}}\sphinxbfcode{\sphinxupquote{volumen\_esfera}}}{\sphinxparam{\DUrole{n}{r}}}{}
\pysigstopsignatures
\sphinxAtStartPar
Calcula el volumen de una esfera dado su radio.
\begin{quote}\begin{description}
\sphinxlineitem{Parámetros}
\sphinxAtStartPar
\sphinxstyleliteralstrong{\sphinxupquote{r}} (\sphinxstyleliteralemphasis{\sphinxupquote{float}}) \textendash{} El radio de la esfera.

\sphinxlineitem{Devuelve}
\sphinxAtStartPar
El volumen de la esfera calculado como (4/3) * π * r\textasciicircum{}3.

\sphinxlineitem{Tipo del valor devuelto}
\sphinxAtStartPar
float

\end{description}\end{quote}

\end{fulllineitems}



\section{Practica 3.1.3}
\label{\detokenize{pr3:module-pr3.1_3}}\label{\detokenize{pr3:practica-3-1-3}}\index{module@\spxentry{module}!pr3.1\_3@\spxentry{pr3.1\_3}}\index{pr3.1\_3@\spxentry{pr3.1\_3}!module@\spxentry{module}}\index{main() (en el módulo pr3.1\_3)@\spxentry{main()}\spxextra{en el módulo pr3.1\_3}}

\begin{fulllineitems}
\phantomsection\label{\detokenize{pr3:pr3.1_3.main}}
\pysigstartsignatures
\pysiglinewithargsret{\sphinxcode{\sphinxupquote{pr3.1\_3.}}\sphinxbfcode{\sphinxupquote{main}}}{}{}
\pysigstopsignatures
\sphinxAtStartPar
Solicita al usuario los coeficientes de una ecuación de primer grado
y muestra su solución.

\end{fulllineitems}

\index{solucion\_ecuacion\_primero\_grado() (en el módulo pr3.1\_3)@\spxentry{solucion\_ecuacion\_primero\_grado()}\spxextra{en el módulo pr3.1\_3}}

\begin{fulllineitems}
\phantomsection\label{\detokenize{pr3:pr3.1_3.solucion_ecuacion_primero_grado}}
\pysigstartsignatures
\pysiglinewithargsret{\sphinxcode{\sphinxupquote{pr3.1\_3.}}\sphinxbfcode{\sphinxupquote{solucion\_ecuacion\_primero\_grado}}}{\sphinxparam{\DUrole{n}{a}}\sphinxparamcomma \sphinxparam{\DUrole{n}{b}}}{}
\pysigstopsignatures
\sphinxAtStartPar
Calcula la solución de una ecuación de primer grado de la forma ax + b = 0.
\begin{quote}\begin{description}
\sphinxlineitem{Parámetros}\begin{itemize}
\item {} 
\sphinxAtStartPar
\sphinxstyleliteralstrong{\sphinxupquote{a}} (\sphinxstyleliteralemphasis{\sphinxupquote{float}}) \textendash{} El coeficiente de la variable x.

\item {} 
\sphinxAtStartPar
\sphinxstyleliteralstrong{\sphinxupquote{b}} (\sphinxstyleliteralemphasis{\sphinxupquote{float}}) \textendash{} El término independiente.

\end{itemize}

\sphinxlineitem{Devuelve}
\sphinxAtStartPar
\begin{description}
\sphinxlineitem{Una cadena que describe la solución:}\begin{itemize}
\item {} 
\sphinxAtStartPar
Si \sphinxtitleref{a != 0}, retorna el valor de x.

\item {} 
\sphinxAtStartPar
Si \sphinxtitleref{a == 0} y \sphinxtitleref{b == 0}, indica que la ecuación tiene infinitas soluciones.

\item {} 
\sphinxAtStartPar
Si \sphinxtitleref{a == 0} y \sphinxtitleref{b != 0}, indica que no hay solución.

\end{itemize}

\end{description}


\sphinxlineitem{Tipo del valor devuelto}
\sphinxAtStartPar
str

\end{description}\end{quote}

\end{fulllineitems}



\section{Practica 3.1.4}
\label{\detokenize{pr3:module-pr3.1_4}}\label{\detokenize{pr3:practica-3-1-4}}\index{module@\spxentry{module}!pr3.1\_4@\spxentry{pr3.1\_4}}\index{pr3.1\_4@\spxentry{pr3.1\_4}!module@\spxentry{module}}\index{main() (en el módulo pr3.1\_4)@\spxentry{main()}\spxextra{en el módulo pr3.1\_4}}

\begin{fulllineitems}
\phantomsection\label{\detokenize{pr3:pr3.1_4.main}}
\pysigstartsignatures
\pysiglinewithargsret{\sphinxcode{\sphinxupquote{pr3.1\_4.}}\sphinxbfcode{\sphinxupquote{main}}}{}{}
\pysigstopsignatures
\sphinxAtStartPar
Solicita al usuario los coeficientes de una ecuación de segundo grado
y muestra las soluciones de la ecuación.

\end{fulllineitems}

\index{resolver\_ecuacion\_segundo\_grado() (en el módulo pr3.1\_4)@\spxentry{resolver\_ecuacion\_segundo\_grado()}\spxextra{en el módulo pr3.1\_4}}

\begin{fulllineitems}
\phantomsection\label{\detokenize{pr3:pr3.1_4.resolver_ecuacion_segundo_grado}}
\pysigstartsignatures
\pysiglinewithargsret{\sphinxcode{\sphinxupquote{pr3.1\_4.}}\sphinxbfcode{\sphinxupquote{resolver\_ecuacion\_segundo\_grado}}}{\sphinxparam{\DUrole{n}{a}}\sphinxparamcomma \sphinxparam{\DUrole{n}{b}}\sphinxparamcomma \sphinxparam{\DUrole{n}{c}}}{}
\pysigstopsignatures
\sphinxAtStartPar
Resuelve una ecuación de segundo grado de la forma ax\textasciicircum{}2 + bx + c = 0.
\begin{quote}\begin{description}
\sphinxlineitem{Parámetros}\begin{itemize}
\item {} 
\sphinxAtStartPar
\sphinxstyleliteralstrong{\sphinxupquote{a}} (\sphinxstyleliteralemphasis{\sphinxupquote{float}}) \textendash{} El coeficiente de x\textasciicircum{}2.

\item {} 
\sphinxAtStartPar
\sphinxstyleliteralstrong{\sphinxupquote{b}} (\sphinxstyleliteralemphasis{\sphinxupquote{float}}) \textendash{} El coeficiente de x.

\item {} 
\sphinxAtStartPar
\sphinxstyleliteralstrong{\sphinxupquote{c}} (\sphinxstyleliteralemphasis{\sphinxupquote{float}}) \textendash{} El término independiente.

\end{itemize}

\sphinxlineitem{Devuelve}
\sphinxAtStartPar
\begin{description}
\sphinxlineitem{Mensaje indicando las soluciones de la ecuación:}\begin{itemize}
\item {} 
\sphinxAtStartPar
Si el discriminante es positivo, retorna las dos soluciones reales.

\item {} 
\sphinxAtStartPar
Si el discriminante es cero, retorna una solución real.

\item {} 
\sphinxAtStartPar
Si el discriminante es negativo, indica que no tiene soluciones reales.

\end{itemize}

\end{description}


\sphinxlineitem{Tipo del valor devuelto}
\sphinxAtStartPar
str

\end{description}\end{quote}

\end{fulllineitems}



\section{Practica 3.1.5}
\label{\detokenize{pr3:module-pr3.1_5}}\label{\detokenize{pr3:practica-3-1-5}}\index{module@\spxentry{module}!pr3.1\_5@\spxentry{pr3.1\_5}}\index{pr3.1\_5@\spxentry{pr3.1\_5}!module@\spxentry{module}}\index{calcular\_porcentaje() (en el módulo pr3.1\_5)@\spxentry{calcular\_porcentaje()}\spxextra{en el módulo pr3.1\_5}}

\begin{fulllineitems}
\phantomsection\label{\detokenize{pr3:pr3.1_5.calcular_porcentaje}}
\pysigstartsignatures
\pysiglinewithargsret{\sphinxcode{\sphinxupquote{pr3.1\_5.}}\sphinxbfcode{\sphinxupquote{calcular\_porcentaje}}}{\sphinxparam{\DUrole{n}{a}}\sphinxparamcomma \sphinxparam{\DUrole{n}{b}}}{}
\pysigstopsignatures
\sphinxAtStartPar
Calcula el porcentaje de una parte con respecto a un total.
\begin{quote}\begin{description}
\sphinxlineitem{Parámetros}\begin{itemize}
\item {} 
\sphinxAtStartPar
\sphinxstyleliteralstrong{\sphinxupquote{a}} (\sphinxstyleliteralemphasis{\sphinxupquote{float}}) \textendash{} El valor de la parte.

\item {} 
\sphinxAtStartPar
\sphinxstyleliteralstrong{\sphinxupquote{b}} (\sphinxstyleliteralemphasis{\sphinxupquote{float}}) \textendash{} El valor del total.

\end{itemize}

\sphinxlineitem{Devuelve}
\sphinxAtStartPar
El porcentaje calculado de la parte con respecto al total.
str: Un mensaje indicando si el cálculo fue posible o no.

\sphinxlineitem{Tipo del valor devuelto}
\sphinxAtStartPar
float

\end{description}\end{quote}

\end{fulllineitems}

\index{main() (en el módulo pr3.1\_5)@\spxentry{main()}\spxextra{en el módulo pr3.1\_5}}

\begin{fulllineitems}
\phantomsection\label{\detokenize{pr3:pr3.1_5.main}}
\pysigstartsignatures
\pysiglinewithargsret{\sphinxcode{\sphinxupquote{pr3.1\_5.}}\sphinxbfcode{\sphinxupquote{main}}}{}{}
\pysigstopsignatures
\sphinxAtStartPar
Solicita al usuario los valores de la parte y el total, y calcula el porcentaje.
Si el total es cero, pide al usuario que ingrese un valor válido.

\end{fulllineitems}



\section{Practica 3.2.1}
\label{\detokenize{pr3:module-pr3.2_1}}\label{\detokenize{pr3:practica-3-2-1}}\index{module@\spxentry{module}!pr3.2\_1@\spxentry{pr3.2\_1}}\index{pr3.2\_1@\spxentry{pr3.2\_1}!module@\spxentry{module}}\index{analizar\_lista\_numeros() (en el módulo pr3.2\_1)@\spxentry{analizar\_lista\_numeros()}\spxextra{en el módulo pr3.2\_1}}

\begin{fulllineitems}
\phantomsection\label{\detokenize{pr3:pr3.2_1.analizar_lista_numeros}}
\pysigstartsignatures
\pysiglinewithargsret{\sphinxcode{\sphinxupquote{pr3.2\_1.}}\sphinxbfcode{\sphinxupquote{analizar\_lista\_numeros}}}{\sphinxparam{\DUrole{n}{numeros}}}{}
\pysigstopsignatures
\sphinxAtStartPar
Analiza una lista de números y devuelve el número mayor y menor de la lista.
\begin{quote}\begin{description}
\sphinxlineitem{Parámetros}
\sphinxAtStartPar
\sphinxstyleliteralstrong{\sphinxupquote{numeros}} (\sphinxstyleliteralemphasis{\sphinxupquote{list}}) \textendash{} Una lista de números enteros.

\sphinxlineitem{Devuelve}
\sphinxAtStartPar
El número mayor y menor de la lista.

\sphinxlineitem{Tipo del valor devuelto}
\sphinxAtStartPar
tuple

\end{description}\end{quote}

\end{fulllineitems}

\index{celsius\_a\_kelvin() (en el módulo pr3.2\_1)@\spxentry{celsius\_a\_kelvin()}\spxextra{en el módulo pr3.2\_1}}

\begin{fulllineitems}
\phantomsection\label{\detokenize{pr3:pr3.2_1.celsius_a_kelvin}}
\pysigstartsignatures
\pysiglinewithargsret{\sphinxcode{\sphinxupquote{pr3.2\_1.}}\sphinxbfcode{\sphinxupquote{celsius\_a\_kelvin}}}{\sphinxparam{\DUrole{n}{celsius}}}{}
\pysigstopsignatures
\sphinxAtStartPar
Convierte grados Celsius a Kelvin.

\sphinxAtStartPar
La fórmula para la conversión es: K = °C + 273
\begin{quote}\begin{description}
\sphinxlineitem{Parámetros}
\sphinxAtStartPar
\sphinxstyleliteralstrong{\sphinxupquote{celsius}} (\sphinxstyleliteralemphasis{\sphinxupquote{float}}) \textendash{} La temperatura en grados Celsius.

\sphinxlineitem{Devuelve}
\sphinxAtStartPar
La temperatura equivalente en grados Kelvin.

\sphinxlineitem{Tipo del valor devuelto}
\sphinxAtStartPar
float

\end{description}\end{quote}

\end{fulllineitems}

\index{es\_primo() (en el módulo pr3.2\_1)@\spxentry{es\_primo()}\spxextra{en el módulo pr3.2\_1}}

\begin{fulllineitems}
\phantomsection\label{\detokenize{pr3:pr3.2_1.es_primo}}
\pysigstartsignatures
\pysiglinewithargsret{\sphinxcode{\sphinxupquote{pr3.2\_1.}}\sphinxbfcode{\sphinxupquote{es\_primo}}}{\sphinxparam{\DUrole{n}{num}}}{}
\pysigstopsignatures
\sphinxAtStartPar
Determina si un número es primo.

\sphinxAtStartPar
Un número primo es aquel que solo es divisible por 1 y por sí mismo.
\begin{quote}\begin{description}
\sphinxlineitem{Parámetros}
\sphinxAtStartPar
\sphinxstyleliteralstrong{\sphinxupquote{num}} (\sphinxstyleliteralemphasis{\sphinxupquote{int}}) \textendash{} El número a verificar.

\sphinxlineitem{Devuelve}
\sphinxAtStartPar
True si el número es primo, False si no lo es.

\sphinxlineitem{Tipo del valor devuelto}
\sphinxAtStartPar
bool

\end{description}\end{quote}

\end{fulllineitems}

\index{main() (en el módulo pr3.2\_1)@\spxentry{main()}\spxextra{en el módulo pr3.2\_1}}

\begin{fulllineitems}
\phantomsection\label{\detokenize{pr3:pr3.2_1.main}}
\pysigstartsignatures
\pysiglinewithargsret{\sphinxcode{\sphinxupquote{pr3.2\_1.}}\sphinxbfcode{\sphinxupquote{main}}}{}{}
\pysigstopsignatures
\sphinxAtStartPar
Función principal que solicita al usuario una serie de entradas y realiza las tareas:
1. Verifica si un número es primo.
2. Convierte una temperatura de grados Celsius a Kelvin.
3. Encuentra el número mayor y menor en una lista de números.

\end{fulllineitems}



\section{Practica 3.2.2}
\label{\detokenize{pr3:module-pr3.2_2}}\label{\detokenize{pr3:practica-3-2-2}}\index{module@\spxentry{module}!pr3.2\_2@\spxentry{pr3.2\_2}}\index{pr3.2\_2@\spxentry{pr3.2\_2}!module@\spxentry{module}}\index{convertir\_numero() (en el módulo pr3.2\_2)@\spxentry{convertir\_numero()}\spxextra{en el módulo pr3.2\_2}}

\begin{fulllineitems}
\phantomsection\label{\detokenize{pr3:pr3.2_2.convertir_numero}}
\pysigstartsignatures
\pysiglinewithargsret{\sphinxcode{\sphinxupquote{pr3.2\_2.}}\sphinxbfcode{\sphinxupquote{convertir\_numero}}}{}{}
\pysigstopsignatures
\sphinxAtStartPar
Convierte un número entero a cadena y viceversa, mostrando los tipos de datos en cada paso.

\sphinxAtStartPar
Solicita al usuario un número entero, lo convierte a cadena, lo muestra con su tipo,
y luego lo convierte nuevamente a entero y muestra el tipo de nuevo.

\end{fulllineitems}

\index{convertir\_y\_calcular\_cuadrado() (en el módulo pr3.2\_2)@\spxentry{convertir\_y\_calcular\_cuadrado()}\spxextra{en el módulo pr3.2\_2}}

\begin{fulllineitems}
\phantomsection\label{\detokenize{pr3:pr3.2_2.convertir_y_calcular_cuadrado}}
\pysigstartsignatures
\pysiglinewithargsret{\sphinxcode{\sphinxupquote{pr3.2\_2.}}\sphinxbfcode{\sphinxupquote{convertir\_y\_calcular\_cuadrado}}}{}{}
\pysigstopsignatures
\sphinxAtStartPar
Convierte un número decimal representado como cadena a tipo flotante,
luego calcula y muestra el cuadrado de ese número.

\sphinxAtStartPar
Solicita al usuario un número decimal como cadena, lo convierte a flotante y calcula su cuadrado.

\end{fulllineitems}

\index{eliminar\_duplicados() (en el módulo pr3.2\_2)@\spxentry{eliminar\_duplicados()}\spxextra{en el módulo pr3.2\_2}}

\begin{fulllineitems}
\phantomsection\label{\detokenize{pr3:pr3.2_2.eliminar_duplicados}}
\pysigstartsignatures
\pysiglinewithargsret{\sphinxcode{\sphinxupquote{pr3.2\_2.}}\sphinxbfcode{\sphinxupquote{eliminar\_duplicados}}}{}{}
\pysigstopsignatures
\sphinxAtStartPar
Convierte una lista de palabras a un conjunto para eliminar duplicados,
y luego vuelve a convertir el conjunto en una lista.

\sphinxAtStartPar
Solicita al usuario una lista de palabras separadas por espacios, elimina los duplicados
convirtiéndola en un conjunto, y luego vuelve a convertir el conjunto a una lista.

\end{fulllineitems}

\index{main() (en el módulo pr3.2\_2)@\spxentry{main()}\spxextra{en el módulo pr3.2\_2}}

\begin{fulllineitems}
\phantomsection\label{\detokenize{pr3:pr3.2_2.main}}
\pysigstartsignatures
\pysiglinewithargsret{\sphinxcode{\sphinxupquote{pr3.2\_2.}}\sphinxbfcode{\sphinxupquote{main}}}{}{}
\pysigstopsignatures
\sphinxAtStartPar
Función principal que ejecuta las tres tareas de conversión de tipos de datos:
1. Convierte un número entero a cadena y viceversa.
2. Convierte un número decimal en cadena a flotante y calcula su cuadrado.
3. Elimina duplicados de una lista de palabras usando un conjunto y luego la convierte de nuevo a lista.

\end{fulllineitems}



\section{Practica 3.3.1}
\label{\detokenize{pr3:module-pr3.3_1}}\label{\detokenize{pr3:practica-3-3-1}}\index{module@\spxentry{module}!pr3.3\_1@\spxentry{pr3.3\_1}}\index{pr3.3\_1@\spxentry{pr3.3\_1}!module@\spxentry{module}}\index{calcular\_cuadrado() (en el módulo pr3.3\_1)@\spxentry{calcular\_cuadrado()}\spxextra{en el módulo pr3.3\_1}}

\begin{fulllineitems}
\phantomsection\label{\detokenize{pr3:pr3.3_1.calcular_cuadrado}}
\pysigstartsignatures
\pysiglinewithargsret{\sphinxcode{\sphinxupquote{pr3.3\_1.}}\sphinxbfcode{\sphinxupquote{calcular\_cuadrado}}}{}{}
\pysigstopsignatures
\sphinxAtStartPar
Solicita al usuario un número real, calcula su cuadrado y verifica si el resultado es un número entero.

\sphinxAtStartPar
El programa calcula el cuadrado de un número real, luego verifica si el resultado es entero y
muestra un mensaje indicando si es entero o no.

\end{fulllineitems}

\index{es\_entero() (en el módulo pr3.3\_1)@\spxentry{es\_entero()}\spxextra{en el módulo pr3.3\_1}}

\begin{fulllineitems}
\phantomsection\label{\detokenize{pr3:pr3.3_1.es_entero}}
\pysigstartsignatures
\pysiglinewithargsret{\sphinxcode{\sphinxupquote{pr3.3\_1.}}\sphinxbfcode{\sphinxupquote{es\_entero}}}{\sphinxparam{\DUrole{n}{numero}}}{}
\pysigstopsignatures
\sphinxAtStartPar
Verifica si un número es entero.
\begin{quote}\begin{description}
\sphinxlineitem{Parámetros}
\sphinxAtStartPar
\sphinxstyleliteralstrong{\sphinxupquote{numero}} (\sphinxstyleliteralemphasis{\sphinxupquote{float}}) \textendash{} El número que se va a verificar.

\sphinxlineitem{Devuelve}
\sphinxAtStartPar
True si el número es entero, False si no lo es.

\sphinxlineitem{Tipo del valor devuelto}
\sphinxAtStartPar
bool

\end{description}\end{quote}

\end{fulllineitems}



\section{Practica 3.3.2}
\label{\detokenize{pr3:module-pr3.3_2}}\label{\detokenize{pr3:practica-3-3-2}}\index{module@\spxentry{module}!pr3.3\_2@\spxentry{pr3.3\_2}}\index{pr3.3\_2@\spxentry{pr3.3\_2}!module@\spxentry{module}}\index{verificar\_condiciones\_logicas() (en el módulo pr3.3\_2)@\spxentry{verificar\_condiciones\_logicas()}\spxextra{en el módulo pr3.3\_2}}

\begin{fulllineitems}
\phantomsection\label{\detokenize{pr3:pr3.3_2.verificar_condiciones_logicas}}
\pysigstartsignatures
\pysiglinewithargsret{\sphinxcode{\sphinxupquote{pr3.3\_2.}}\sphinxbfcode{\sphinxupquote{verificar\_condiciones\_logicas}}}{}{}
\pysigstopsignatures
\sphinxAtStartPar
Realiza dos verificaciones lógicas utilizando operadores “and” y “or”.

\sphinxAtStartPar
Primero, evalúa si ambas condiciones son verdaderas usando el operador “and”:
\sphinxhyphen{} Si “a” es mayor que 2 y “b” es menor que 15.

\sphinxAtStartPar
Luego, evalúa si al menos una de las condiciones es verdadera usando el operador “or”:
\sphinxhyphen{} Si el usuario es premium o tiene un cupón.

\sphinxAtStartPar
Los resultados de ambas verificaciones se imprimen por separado.

\end{fulllineitems}


\sphinxstepscope


\chapter{Practica 4}
\label{\detokenize{pr4:practica-4}}\label{\detokenize{pr4::doc}}

\section{Practica 4.1}
\label{\detokenize{pr4:module-pr4.1}}\label{\detokenize{pr4:practica-4-1}}\index{module@\spxentry{module}!pr4.1@\spxentry{pr4.1}}\index{pr4.1@\spxentry{pr4.1}!module@\spxentry{module}}\index{calcular\_perimetro\_area\_cuadrado() (en el módulo pr4.1)@\spxentry{calcular\_perimetro\_area\_cuadrado()}\spxextra{en el módulo pr4.1}}

\begin{fulllineitems}
\phantomsection\label{\detokenize{pr4:pr4.1.calcular_perimetro_area_cuadrado}}
\pysigstartsignatures
\pysiglinewithargsret{\sphinxcode{\sphinxupquote{pr4.1.}}\sphinxbfcode{\sphinxupquote{calcular\_perimetro\_area\_cuadrado}}}{\sphinxparam{\DUrole{n}{lado}}}{}
\pysigstopsignatures
\sphinxAtStartPar
Calcula el perímetro y el área de un cuadrado dado el valor de su lado.
\begin{quote}\begin{description}
\sphinxlineitem{Parámetros}
\sphinxAtStartPar
\sphinxstyleliteralstrong{\sphinxupquote{lado}} (\sphinxstyleliteralemphasis{\sphinxupquote{float}}) \textendash{} El valor del lado del cuadrado.

\sphinxlineitem{Devuelve}
\sphinxAtStartPar
\begin{description}
\sphinxlineitem{Un tuple que contiene dos valores:}\begin{itemize}
\item {} 
\sphinxAtStartPar
El perímetro del cuadrado (float).

\item {} 
\sphinxAtStartPar
El área del cuadrado (float).

\end{itemize}

\end{description}


\sphinxlineitem{Tipo del valor devuelto}
\sphinxAtStartPar
tuple

\end{description}\end{quote}

\end{fulllineitems}



\section{Practica 4.2}
\label{\detokenize{pr4:module-pr4.2}}\label{\detokenize{pr4:practica-4-2}}\index{module@\spxentry{module}!pr4.2@\spxentry{pr4.2}}\index{pr4.2@\spxentry{pr4.2}!module@\spxentry{module}}\index{calcular\_perimetro\_area\_rectangulo() (en el módulo pr4.2)@\spxentry{calcular\_perimetro\_area\_rectangulo()}\spxextra{en el módulo pr4.2}}

\begin{fulllineitems}
\phantomsection\label{\detokenize{pr4:pr4.2.calcular_perimetro_area_rectangulo}}
\pysigstartsignatures
\pysiglinewithargsret{\sphinxcode{\sphinxupquote{pr4.2.}}\sphinxbfcode{\sphinxupquote{calcular\_perimetro\_area\_rectangulo}}}{\sphinxparam{\DUrole{n}{lado1}}\sphinxparamcomma \sphinxparam{\DUrole{n}{lado2}}}{}
\pysigstopsignatures
\sphinxAtStartPar
Calcula el perímetro y el área de un rectángulo dado el valor de sus dos lados.
\begin{quote}\begin{description}
\sphinxlineitem{Parámetros}\begin{itemize}
\item {} 
\sphinxAtStartPar
\sphinxstyleliteralstrong{\sphinxupquote{lado1}} (\sphinxstyleliteralemphasis{\sphinxupquote{float}}) \textendash{} El valor del primer lado del rectángulo.

\item {} 
\sphinxAtStartPar
\sphinxstyleliteralstrong{\sphinxupquote{lado2}} (\sphinxstyleliteralemphasis{\sphinxupquote{float}}) \textendash{} El valor del segundo lado del rectángulo.

\end{itemize}

\sphinxlineitem{Devuelve}
\sphinxAtStartPar
\begin{description}
\sphinxlineitem{Un tuple que contiene dos valores:}\begin{itemize}
\item {} 
\sphinxAtStartPar
El perímetro del rectángulo (float).

\item {} 
\sphinxAtStartPar
El área del rectángulo (float).

\end{itemize}

\end{description}


\sphinxlineitem{Tipo del valor devuelto}
\sphinxAtStartPar
tuple

\end{description}\end{quote}

\end{fulllineitems}



\section{Practica 4.3}
\label{\detokenize{pr4:module-pr4.3}}\label{\detokenize{pr4:practica-4-3}}\index{module@\spxentry{module}!pr4.3@\spxentry{pr4.3}}\index{pr4.3@\spxentry{pr4.3}!module@\spxentry{module}}\index{calcular\_area\_triangulo() (en el módulo pr4.3)@\spxentry{calcular\_area\_triangulo()}\spxextra{en el módulo pr4.3}}

\begin{fulllineitems}
\phantomsection\label{\detokenize{pr4:pr4.3.calcular_area_triangulo}}
\pysigstartsignatures
\pysiglinewithargsret{\sphinxcode{\sphinxupquote{pr4.3.}}\sphinxbfcode{\sphinxupquote{calcular\_area\_triangulo}}}{\sphinxparam{\DUrole{n}{base}}\sphinxparamcomma \sphinxparam{\DUrole{n}{altura}}}{}
\pysigstopsignatures
\sphinxAtStartPar
Calcula el área de un triángulo dado su base y altura.
\begin{quote}\begin{description}
\sphinxlineitem{Parámetros}\begin{itemize}
\item {} 
\sphinxAtStartPar
\sphinxstyleliteralstrong{\sphinxupquote{base}} (\sphinxstyleliteralemphasis{\sphinxupquote{float}}) \textendash{} El valor de la base del triángulo.

\item {} 
\sphinxAtStartPar
\sphinxstyleliteralstrong{\sphinxupquote{altura}} (\sphinxstyleliteralemphasis{\sphinxupquote{float}}) \textendash{} El valor de la altura del triángulo.

\end{itemize}

\sphinxlineitem{Devuelve}
\sphinxAtStartPar
El área del triángulo calculada con la fórmula (base * altura) / 2.

\sphinxlineitem{Tipo del valor devuelto}
\sphinxAtStartPar
float

\end{description}\end{quote}

\end{fulllineitems}



\section{Practica 4.4}
\label{\detokenize{pr4:module-pr4.4}}\label{\detokenize{pr4:practica-4-4}}\index{module@\spxentry{module}!pr4.4@\spxentry{pr4.4}}\index{pr4.4@\spxentry{pr4.4}!module@\spxentry{module}}\index{calcular\_area\_perimetro\_triangulo() (en el módulo pr4.4)@\spxentry{calcular\_area\_perimetro\_triangulo()}\spxextra{en el módulo pr4.4}}

\begin{fulllineitems}
\phantomsection\label{\detokenize{pr4:pr4.4.calcular_area_perimetro_triangulo}}
\pysigstartsignatures
\pysiglinewithargsret{\sphinxcode{\sphinxupquote{pr4.4.}}\sphinxbfcode{\sphinxupquote{calcular\_area\_perimetro\_triangulo}}}{\sphinxparam{\DUrole{n}{lado1}}\sphinxparamcomma \sphinxparam{\DUrole{n}{lado2}}\sphinxparamcomma \sphinxparam{\DUrole{n}{lado3}}}{}
\pysigstopsignatures
\sphinxAtStartPar
Calcula el área y el perímetro de un triángulo dado sus tres lados utilizando la fórmula de Herón para el área.
\begin{quote}\begin{description}
\sphinxlineitem{Parámetros}\begin{itemize}
\item {} 
\sphinxAtStartPar
\sphinxstyleliteralstrong{\sphinxupquote{lado1}} (\sphinxstyleliteralemphasis{\sphinxupquote{float}}) \textendash{} El valor del primer lado del triángulo.

\item {} 
\sphinxAtStartPar
\sphinxstyleliteralstrong{\sphinxupquote{lado2}} (\sphinxstyleliteralemphasis{\sphinxupquote{float}}) \textendash{} El valor del segundo lado del triángulo.

\item {} 
\sphinxAtStartPar
\sphinxstyleliteralstrong{\sphinxupquote{lado3}} (\sphinxstyleliteralemphasis{\sphinxupquote{float}}) \textendash{} El valor del tercer lado del triángulo.

\end{itemize}

\sphinxlineitem{Devuelve}
\sphinxAtStartPar
\begin{description}
\sphinxlineitem{Un tuple que contiene dos valores:}\begin{itemize}
\item {} 
\sphinxAtStartPar
El perímetro del triángulo (float).

\item {} 
\sphinxAtStartPar
El área del triángulo (float).

\end{itemize}

\end{description}


\sphinxlineitem{Tipo del valor devuelto}
\sphinxAtStartPar
tuple

\end{description}\end{quote}

\end{fulllineitems}



\section{Practica 4.5}
\label{\detokenize{pr4:module-pr4.5}}\label{\detokenize{pr4:practica-4-5}}\index{module@\spxentry{module}!pr4.5@\spxentry{pr4.5}}\index{pr4.5@\spxentry{pr4.5}!module@\spxentry{module}}\index{registrar\_persona() (en el módulo pr4.5)@\spxentry{registrar\_persona()}\spxextra{en el módulo pr4.5}}

\begin{fulllineitems}
\phantomsection\label{\detokenize{pr4:pr4.5.registrar_persona}}
\pysigstartsignatures
\pysiglinewithargsret{\sphinxcode{\sphinxupquote{pr4.5.}}\sphinxbfcode{\sphinxupquote{registrar\_persona}}}{}{}
\pysigstopsignatures
\sphinxAtStartPar
Solicita los datos de una persona, como nombre, apellidos, edad y estatura,
y luego muestra la información ingresada.

\sphinxAtStartPar
La función no toma parámetros ni devuelve valores. Solicita la entrada del
usuario a través de la consola y muestra los resultados en formato legible.
\begin{description}
\sphinxlineitem{Ejemplo de ejecución:}
\sphinxAtStartPar
Introduce tu nombre: John
Introduce tus apellidos: Doe
Introduce tu edad: 30
Introduce tu estatura (en metros): 1.75

\sphinxlineitem{Salida:}
\sphinxAtStartPar
Información de la persona:
Nombre: John
Apellidos: Doe
Edad: 30 años
Estatura: 1.75 metros

\end{description}

\end{fulllineitems}



\section{Practica 4.6}
\label{\detokenize{pr4:module-pr4.6}}\label{\detokenize{pr4:practica-4-6}}\index{module@\spxentry{module}!pr4.6@\spxentry{pr4.6}}\index{pr4.6@\spxentry{pr4.6}!module@\spxentry{module}}\index{calcular\_area\_trapecio() (en el módulo pr4.6)@\spxentry{calcular\_area\_trapecio()}\spxextra{en el módulo pr4.6}}

\begin{fulllineitems}
\phantomsection\label{\detokenize{pr4:pr4.6.calcular_area_trapecio}}
\pysigstartsignatures
\pysiglinewithargsret{\sphinxcode{\sphinxupquote{pr4.6.}}\sphinxbfcode{\sphinxupquote{calcular\_area\_trapecio}}}{\sphinxparam{\DUrole{n}{base\_mayor}}\sphinxparamcomma \sphinxparam{\DUrole{n}{base\_menor}}\sphinxparamcomma \sphinxparam{\DUrole{n}{altura}}}{}
\pysigstopsignatures
\sphinxAtStartPar
Calcula el área de un trapecio dado su base mayor, base menor y altura.
\begin{description}
\sphinxlineitem{La fórmula utilizada es:}
\sphinxAtStartPar
Área = ((base mayor + base menor) * altura) / 2

\end{description}
\begin{quote}\begin{description}
\sphinxlineitem{Parámetros}\begin{itemize}
\item {} 
\sphinxAtStartPar
\sphinxstyleliteralstrong{\sphinxupquote{base\_mayor}} (\sphinxstyleliteralemphasis{\sphinxupquote{float}}) \textendash{} La longitud de la base mayor del trapecio.

\item {} 
\sphinxAtStartPar
\sphinxstyleliteralstrong{\sphinxupquote{base\_menor}} (\sphinxstyleliteralemphasis{\sphinxupquote{float}}) \textendash{} La longitud de la base menor del trapecio.

\item {} 
\sphinxAtStartPar
\sphinxstyleliteralstrong{\sphinxupquote{altura}} (\sphinxstyleliteralemphasis{\sphinxupquote{float}}) \textendash{} La altura del trapecio.

\end{itemize}

\sphinxlineitem{Devuelve}
\sphinxAtStartPar
El área del trapecio.

\sphinxlineitem{Tipo del valor devuelto}
\sphinxAtStartPar
float

\end{description}\end{quote}

\end{fulllineitems}



\section{Practica 4.7}
\label{\detokenize{pr4:module-pr4.7}}\label{\detokenize{pr4:practica-4-7}}\index{module@\spxentry{module}!pr4.7@\spxentry{pr4.7}}\index{pr4.7@\spxentry{pr4.7}!module@\spxentry{module}}\index{intercambiar\_valores() (en el módulo pr4.7)@\spxentry{intercambiar\_valores()}\spxextra{en el módulo pr4.7}}

\begin{fulllineitems}
\phantomsection\label{\detokenize{pr4:pr4.7.intercambiar_valores}}
\pysigstartsignatures
\pysiglinewithargsret{\sphinxcode{\sphinxupquote{pr4.7.}}\sphinxbfcode{\sphinxupquote{intercambiar\_valores}}}{\sphinxparam{\DUrole{n}{a}}\sphinxparamcomma \sphinxparam{\DUrole{n}{b}}}{}
\pysigstopsignatures
\sphinxAtStartPar
Intercambia los valores de dos variables.

\sphinxAtStartPar
La función toma dos números, intercambia sus valores y los devuelve.
\begin{quote}\begin{description}
\sphinxlineitem{Parámetros}\begin{itemize}
\item {} 
\sphinxAtStartPar
\sphinxstyleliteralstrong{\sphinxupquote{a}} (\sphinxstyleliteralemphasis{\sphinxupquote{float}}) \textendash{} El primer número.

\item {} 
\sphinxAtStartPar
\sphinxstyleliteralstrong{\sphinxupquote{b}} (\sphinxstyleliteralemphasis{\sphinxupquote{float}}) \textendash{} El segundo número.

\end{itemize}

\sphinxlineitem{Devuelve}
\sphinxAtStartPar
Una tupla con los valores de “b” y “a” intercambiados.

\sphinxlineitem{Tipo del valor devuelto}
\sphinxAtStartPar
tuple

\end{description}\end{quote}

\end{fulllineitems}



\section{Practica 4.8}
\label{\detokenize{pr4:module-pr4.8}}\label{\detokenize{pr4:practica-4-8}}\index{module@\spxentry{module}!pr4.8@\spxentry{pr4.8}}\index{pr4.8@\spxentry{pr4.8}!module@\spxentry{module}}\index{desglosar\_segundos() (en el módulo pr4.8)@\spxentry{desglosar\_segundos()}\spxextra{en el módulo pr4.8}}

\begin{fulllineitems}
\phantomsection\label{\detokenize{pr4:pr4.8.desglosar_segundos}}
\pysigstartsignatures
\pysiglinewithargsret{\sphinxcode{\sphinxupquote{pr4.8.}}\sphinxbfcode{\sphinxupquote{desglosar\_segundos}}}{\sphinxparam{\DUrole{n}{segundos\_totales}}}{}
\pysigstopsignatures
\sphinxAtStartPar
Desglosa una cantidad de segundos en su equivalente en horas, minutos y segundos.

\sphinxAtStartPar
La función toma una cantidad total de segundos y devuelve su equivalente en
horas, minutos y segundos utilizando división entera y el operador módulo.
\begin{quote}\begin{description}
\sphinxlineitem{Parámetros}
\sphinxAtStartPar
\sphinxstyleliteralstrong{\sphinxupquote{segundos\_totales}} (\sphinxstyleliteralemphasis{\sphinxupquote{int}}) \textendash{} La cantidad de segundos a desglosar.

\sphinxlineitem{Devuelve}
\sphinxAtStartPar
Una tupla con las horas, minutos y segundos correspondientes.

\sphinxlineitem{Tipo del valor devuelto}
\sphinxAtStartPar
tuple

\end{description}\end{quote}

\end{fulllineitems}



\section{Practica 4.9}
\label{\detokenize{pr4:module-pr4.9}}\label{\detokenize{pr4:practica-4-9}}\index{module@\spxentry{module}!pr4.9@\spxentry{pr4.9}}\index{pr4.9@\spxentry{pr4.9}!module@\spxentry{module}}\index{desglosar\_segundos() (en el módulo pr4.9)@\spxentry{desglosar\_segundos()}\spxextra{en el módulo pr4.9}}

\begin{fulllineitems}
\phantomsection\label{\detokenize{pr4:pr4.9.desglosar_segundos}}
\pysigstartsignatures
\pysiglinewithargsret{\sphinxcode{\sphinxupquote{pr4.9.}}\sphinxbfcode{\sphinxupquote{desglosar\_segundos}}}{\sphinxparam{\DUrole{n}{segundos\_totales}}}{}
\pysigstopsignatures
\sphinxAtStartPar
Desglosa una cantidad de segundos en su equivalente en horas, minutos y segundos.

\sphinxAtStartPar
La función toma una cantidad total de segundos y devuelve su equivalente en
horas, minutos y segundos utilizando división entera y el operador módulo.
\begin{quote}\begin{description}
\sphinxlineitem{Parámetros}
\sphinxAtStartPar
\sphinxstyleliteralstrong{\sphinxupquote{segundos\_totales}} (\sphinxstyleliteralemphasis{\sphinxupquote{int}}) \textendash{} La cantidad de segundos a desglosar.

\sphinxlineitem{Devuelve}
\sphinxAtStartPar
Una tupla con las horas, minutos y segundos correspondientes.

\sphinxlineitem{Tipo del valor devuelto}
\sphinxAtStartPar
tuple

\end{description}\end{quote}

\end{fulllineitems}



\section{Practica 4.10}
\label{\detokenize{pr4:module-pr4.10}}\label{\detokenize{pr4:practica-4-10}}\index{module@\spxentry{module}!pr4.10@\spxentry{pr4.10}}\index{pr4.10@\spxentry{pr4.10}!module@\spxentry{module}}\index{convertir\_a\_pesetas() (en el módulo pr4.10)@\spxentry{convertir\_a\_pesetas()}\spxextra{en el módulo pr4.10}}

\begin{fulllineitems}
\phantomsection\label{\detokenize{pr4:pr4.10.convertir_a_pesetas}}
\pysigstartsignatures
\pysiglinewithargsret{\sphinxcode{\sphinxupquote{pr4.10.}}\sphinxbfcode{\sphinxupquote{convertir\_a\_pesetas}}}{\sphinxparam{\DUrole{n}{euros}}}{}
\pysigstopsignatures
\sphinxAtStartPar
Convierte una cantidad en euros a pesetas.

\sphinxAtStartPar
Esta función toma una cantidad en euros y la convierte a pesetas utilizando
el tipo de cambio fijo de 1 euro = 166386 pesetas.
\begin{quote}\begin{description}
\sphinxlineitem{Parámetros}
\sphinxAtStartPar
\sphinxstyleliteralstrong{\sphinxupquote{euros}} (\sphinxstyleliteralemphasis{\sphinxupquote{float}}) \textendash{} La cantidad en euros a convertir.

\sphinxlineitem{Devuelve}
\sphinxAtStartPar
La cantidad equivalente en pesetas.

\sphinxlineitem{Tipo del valor devuelto}
\sphinxAtStartPar
float

\end{description}\end{quote}

\end{fulllineitems}


\sphinxstepscope


\chapter{Practica 5}
\label{\detokenize{pr5:practica-5}}\label{\detokenize{pr5::doc}}

\section{Practica 5.1}
\label{\detokenize{pr5:module-pr5.1}}\label{\detokenize{pr5:practica-5-1}}\index{module@\spxentry{module}!pr5.1@\spxentry{pr5.1}}\index{pr5.1@\spxentry{pr5.1}!module@\spxentry{module}}\index{dividir() (en el módulo pr5.1)@\spxentry{dividir()}\spxextra{en el módulo pr5.1}}

\begin{fulllineitems}
\phantomsection\label{\detokenize{pr5:pr5.1.dividir}}
\pysigstartsignatures
\pysiglinewithargsret{\sphinxcode{\sphinxupquote{pr5.1.}}\sphinxbfcode{\sphinxupquote{dividir}}}{\sphinxparam{\DUrole{n}{dividendo}}\sphinxparamcomma \sphinxparam{\DUrole{n}{divisor}}}{}
\pysigstopsignatures
\sphinxAtStartPar
Realiza la división de dos números, verificando que el divisor no sea cero.

\sphinxAtStartPar
Esta función recibe dos números: el dividendo y el divisor, y devuelve el
resultado de la división. Si el divisor es cero, la función devuelve
un mensaje indicando que no es posible realizar la división.
\begin{quote}\begin{description}
\sphinxlineitem{Parámetros}\begin{itemize}
\item {} 
\sphinxAtStartPar
\sphinxstyleliteralstrong{\sphinxupquote{dividendo}} (\sphinxstyleliteralemphasis{\sphinxupquote{float}}) \textendash{} El número que será dividido.

\item {} 
\sphinxAtStartPar
\sphinxstyleliteralstrong{\sphinxupquote{divisor}} (\sphinxstyleliteralemphasis{\sphinxupquote{float}}) \textendash{} El número que divide al dividendo.

\end{itemize}

\sphinxlineitem{Devuelve}
\sphinxAtStartPar
\begin{description}
\sphinxlineitem{El resultado de la división si el divisor es diferente de cero,}
\sphinxAtStartPar
o un mensaje de error si el divisor es cero.

\end{description}


\sphinxlineitem{Tipo del valor devuelto}
\sphinxAtStartPar
float or str

\end{description}\end{quote}

\end{fulllineitems}



\section{Practica 5.2}
\label{\detokenize{pr5:module-pr5.2}}\label{\detokenize{pr5:practica-5-2}}\index{module@\spxentry{module}!pr5.2@\spxentry{pr5.2}}\index{pr5.2@\spxentry{pr5.2}!module@\spxentry{module}}\index{evaluar\_beneficios() (en el módulo pr5.2)@\spxentry{evaluar\_beneficios()}\spxextra{en el módulo pr5.2}}

\begin{fulllineitems}
\phantomsection\label{\detokenize{pr5:pr5.2.evaluar_beneficios}}
\pysigstartsignatures
\pysiglinewithargsret{\sphinxcode{\sphinxupquote{pr5.2.}}\sphinxbfcode{\sphinxupquote{evaluar\_beneficios}}}{\sphinxparam{\DUrole{n}{ingresos}\DUrole{p}{:}\DUrole{w}{ }\DUrole{n}{int}}\sphinxparamcomma \sphinxparam{\DUrole{n}{gastos}\DUrole{p}{:}\DUrole{w}{ }\DUrole{n}{int}}}{{ $\rightarrow$ int}}
\pysigstopsignatures
\sphinxAtStartPar
Calcula los beneficios de una empresa a partir de sus ingresos y gastos.

\sphinxAtStartPar
Esta función evalúa si los beneficios de una empresa son positivos, negativos o nulos,
y devuelve un valor entero dependiendo de la situación de la empresa.
\begin{itemize}
\item {} 
\sphinxAtStartPar
Si los beneficios son positivos, imprime «La empresa es solvente» y devuelve +1.

\item {} 
\sphinxAtStartPar
Si los beneficios son cero, imprime «Se ha alcanzado el punto de equilibrio» y devuelve 0.

\item {} 
\sphinxAtStartPar
Si los beneficios son negativos, imprime «La empresa esta en numeros rojos» y devuelve \sphinxhyphen{}1.

\end{itemize}
\begin{quote}\begin{description}
\sphinxlineitem{Parámetros}\begin{itemize}
\item {} 
\sphinxAtStartPar
\sphinxstyleliteralstrong{\sphinxupquote{ingresos}} (\sphinxstyleliteralemphasis{\sphinxupquote{int}}) \textendash{} Los ingresos de la empresa.

\item {} 
\sphinxAtStartPar
\sphinxstyleliteralstrong{\sphinxupquote{gastos}} (\sphinxstyleliteralemphasis{\sphinxupquote{int}}) \textendash{} Los gastos de la empresa.

\end{itemize}

\sphinxlineitem{Devuelve}
\sphinxAtStartPar
+1 si la empresa es solvente (beneficios positivos).
0 si se ha alcanzado el punto de equilibrio (beneficios cero).
\sphinxhyphen{}1 si la empresa está en números rojos (beneficios negativos).

\sphinxlineitem{Tipo del valor devuelto}
\sphinxAtStartPar
int

\end{description}\end{quote}

\end{fulllineitems}



\section{Practica 5.3}
\label{\detokenize{pr5:module-pr5.3}}\label{\detokenize{pr5:practica-5-3}}\index{module@\spxentry{module}!pr5.3@\spxentry{pr5.3}}\index{pr5.3@\spxentry{pr5.3}!module@\spxentry{module}}\index{deteccion() (en el módulo pr5.3)@\spxentry{deteccion()}\spxextra{en el módulo pr5.3}}

\begin{fulllineitems}
\phantomsection\label{\detokenize{pr5:pr5.3.deteccion}}
\pysigstartsignatures
\pysiglinewithargsret{\sphinxcode{\sphinxupquote{pr5.3.}}\sphinxbfcode{\sphinxupquote{deteccion}}}{\sphinxparam{\DUrole{n}{a}\DUrole{p}{:}\DUrole{w}{ }\DUrole{n}{int}}\sphinxparamcomma \sphinxparam{\DUrole{n}{b}\DUrole{p}{:}\DUrole{w}{ }\DUrole{n}{int}}\sphinxparamcomma \sphinxparam{\DUrole{n}{c}\DUrole{p}{:}\DUrole{w}{ }\DUrole{n}{int}}}{{ $\rightarrow$ bool}}
\pysigstopsignatures
\sphinxAtStartPar
Detecta si tres números se han introducido en orden creciente.

\sphinxAtStartPar
La función evalúa si los tres números dados están en orden creciente,
es decir, si el primero es menor que el segundo y el segundo es menor que el tercero.
\begin{quote}\begin{description}
\sphinxlineitem{Parámetros}\begin{itemize}
\item {} 
\sphinxAtStartPar
\sphinxstyleliteralstrong{\sphinxupquote{a}} (\sphinxstyleliteralemphasis{\sphinxupquote{int}}) \textendash{} Primer número.

\item {} 
\sphinxAtStartPar
\sphinxstyleliteralstrong{\sphinxupquote{b}} (\sphinxstyleliteralemphasis{\sphinxupquote{int}}) \textendash{} Segundo número.

\item {} 
\sphinxAtStartPar
\sphinxstyleliteralstrong{\sphinxupquote{c}} (\sphinxstyleliteralemphasis{\sphinxupquote{int}}) \textendash{} Tercer número.

\end{itemize}

\sphinxlineitem{Devuelve}
\sphinxAtStartPar
True si los números están en orden creciente.
False si no están en orden creciente.

\sphinxlineitem{Tipo del valor devuelto}
\sphinxAtStartPar
bool

\end{description}\end{quote}

\end{fulllineitems}



\section{Practica 5.4}
\label{\detokenize{pr5:module-pr5.4}}\label{\detokenize{pr5:practica-5-4}}\index{module@\spxentry{module}!pr5.4@\spxentry{pr5.4}}\index{pr5.4@\spxentry{pr5.4}!module@\spxentry{module}}\index{dia\_de\_la\_semana() (en el módulo pr5.4)@\spxentry{dia\_de\_la\_semana()}\spxextra{en el módulo pr5.4}}

\begin{fulllineitems}
\phantomsection\label{\detokenize{pr5:pr5.4.dia_de_la_semana}}
\pysigstartsignatures
\pysiglinewithargsret{\sphinxcode{\sphinxupquote{pr5.4.}}\sphinxbfcode{\sphinxupquote{dia\_de\_la\_semana}}}{\sphinxparam{\DUrole{n}{dia}\DUrole{p}{:}\DUrole{w}{ }\DUrole{n}{int}}}{{ $\rightarrow$ str}}
\pysigstopsignatures
\sphinxAtStartPar
Dada una entrada numérica entre 1 y 7, devuelve el correspondiente día de la semana.

\sphinxAtStartPar
La función recibe un número entero entre 1 y 7 e imprime el nombre del día de la semana correspondiente.
\begin{quote}\begin{description}
\sphinxlineitem{Parámetros}
\sphinxAtStartPar
\sphinxstyleliteralstrong{\sphinxupquote{dia}} (\sphinxstyleliteralemphasis{\sphinxupquote{int}}) \textendash{} Un número entero entre 1 y 7, donde:
1 = lunes, 2 = martes, …, 7 = domingo.

\sphinxlineitem{Devuelve}
\sphinxAtStartPar
El nombre del día de la semana correspondiente al número proporcionado.

\sphinxlineitem{Tipo del valor devuelto}
\sphinxAtStartPar
str

\sphinxlineitem{Muestra}
\sphinxAtStartPar
\sphinxstyleliteralstrong{\sphinxupquote{ValueError}} \textendash{} Si el número ingresado no está entre 1 y 7.

\end{description}\end{quote}

\end{fulllineitems}



\section{Practica 5.5}
\label{\detokenize{pr5:module-pr5.5}}\label{\detokenize{pr5:practica-5-5}}\index{module@\spxentry{module}!pr5.5@\spxentry{pr5.5}}\index{pr5.5@\spxentry{pr5.5}!module@\spxentry{module}}\index{menu\_principal() (en el módulo pr5.5)@\spxentry{menu\_principal()}\spxextra{en el módulo pr5.5}}

\begin{fulllineitems}
\phantomsection\label{\detokenize{pr5:pr5.5.menu_principal}}
\pysigstartsignatures
\pysiglinewithargsret{\sphinxcode{\sphinxupquote{pr5.5.}}\sphinxbfcode{\sphinxupquote{menu\_principal}}}{}{}
\pysigstopsignatures
\sphinxAtStartPar
Muestra un menú interactivo al usuario con diferentes opciones. Según la opción seleccionada,
se muestra un mensaje indicando la acción seleccionada o un mensaje de error si la opción no es válida.

\sphinxAtStartPar
El menú tiene las siguientes opciones:
1. Cargar fichero de datos
2. Almacenar fichero de datos
3. Modificar datos
4. Salir

\sphinxAtStartPar
El programa seguirá mostrando el menú hasta que el usuario elija la opción “Salir”.

\end{fulllineitems}



\section{Practica 5.6}
\label{\detokenize{pr5:module-pr5.6}}\label{\detokenize{pr5:practica-5-6}}\index{module@\spxentry{module}!pr5.6@\spxentry{pr5.6}}\index{pr5.6@\spxentry{pr5.6}!module@\spxentry{module}}\index{tabla\_multiplicar() (en el módulo pr5.6)@\spxentry{tabla\_multiplicar()}\spxextra{en el módulo pr5.6}}

\begin{fulllineitems}
\phantomsection\label{\detokenize{pr5:pr5.6.tabla_multiplicar}}
\pysigstartsignatures
\pysiglinewithargsret{\sphinxcode{\sphinxupquote{pr5.6.}}\sphinxbfcode{\sphinxupquote{tabla\_multiplicar}}}{\sphinxparam{\DUrole{n}{numero}}}{}
\pysigstopsignatures
\sphinxAtStartPar
Muestra la tabla de multiplicar de un número natural dado.

\sphinxAtStartPar
Parámetros:
numero (int): El número natural cuya tabla de multiplicar se va a mostrar.

\sphinxAtStartPar
Si el número dado es negativo, se muestra un mensaje indicando que debe ser un número natural.
De lo contrario, muestra la tabla de multiplicar del número proporcionado, desde 1 hasta 10.

\end{fulllineitems}



\section{Practica 5.7}
\label{\detokenize{pr5:module-pr5.7}}\label{\detokenize{pr5:practica-5-7}}\index{module@\spxentry{module}!pr5.7@\spxentry{pr5.7}}\index{pr5.7@\spxentry{pr5.7}!module@\spxentry{module}}\index{sumar\_numeros\_entre() (en el módulo pr5.7)@\spxentry{sumar\_numeros\_entre()}\spxextra{en el módulo pr5.7}}

\begin{fulllineitems}
\phantomsection\label{\detokenize{pr5:pr5.7.sumar_numeros_entre}}
\pysigstartsignatures
\pysiglinewithargsret{\sphinxcode{\sphinxupquote{pr5.7.}}\sphinxbfcode{\sphinxupquote{sumar\_numeros\_entre}}}{\sphinxparam{\DUrole{n}{inicio}}\sphinxparamcomma \sphinxparam{\DUrole{n}{fin}}\sphinxparamcomma \sphinxparam{\DUrole{n}{metodo}}}{}
\pysigstopsignatures
\sphinxAtStartPar
Calcula la suma de los números enteros entre dos números dados, utilizando un bucle “for” o “while”.

\sphinxAtStartPar
Parámetros:
inicio (int): El número inicial.
fin (int): El número final.
metodo (str): El método a utilizar para la suma (“for” o “while”).

\sphinxAtStartPar
Retorna:
int: La suma de los números entre “inicio” y “fin” inclusive, según el método seleccionado.

\end{fulllineitems}



\section{Practica 5.8}
\label{\detokenize{pr5:module-pr5.8}}\label{\detokenize{pr5:practica-5-8}}\index{module@\spxentry{module}!pr5.8@\spxentry{pr5.8}}\index{pr5.8@\spxentry{pr5.8}!module@\spxentry{module}}\index{calcular\_factorial() (en el módulo pr5.8)@\spxentry{calcular\_factorial()}\spxextra{en el módulo pr5.8}}

\begin{fulllineitems}
\phantomsection\label{\detokenize{pr5:pr5.8.calcular_factorial}}
\pysigstartsignatures
\pysiglinewithargsret{\sphinxcode{\sphinxupquote{pr5.8.}}\sphinxbfcode{\sphinxupquote{calcular\_factorial}}}{\sphinxparam{\DUrole{n}{n}}}{}
\pysigstopsignatures
\sphinxAtStartPar
Calcula el factorial de un número entero no negativo.

\sphinxAtStartPar
Parámetros:
n (int): Número entero no negativo para calcular su factorial.

\sphinxAtStartPar
Retorna:
int: El factorial de \sphinxtitleref{n}.

\sphinxAtStartPar
Levanta:
ValueError: Si \sphinxtitleref{n} es un número negativo.

\end{fulllineitems}



\section{Practica 5.9}
\label{\detokenize{pr5:module-pr5.9}}\label{\detokenize{pr5:practica-5-9}}\index{module@\spxentry{module}!pr5.9@\spxentry{pr5.9}}\index{pr5.9@\spxentry{pr5.9}!module@\spxentry{module}}\index{estadisticas\_numeros() (en el módulo pr5.9)@\spxentry{estadisticas\_numeros()}\spxextra{en el módulo pr5.9}}

\begin{fulllineitems}
\phantomsection\label{\detokenize{pr5:pr5.9.estadisticas_numeros}}
\pysigstartsignatures
\pysiglinewithargsret{\sphinxcode{\sphinxupquote{pr5.9.}}\sphinxbfcode{\sphinxupquote{estadisticas\_numeros}}}{}{}
\pysigstopsignatures
\sphinxAtStartPar
Solicita al usuario una serie de números y calcula las estadísticas más relevantes:
el número más grande, el más pequeño y la media.

\sphinxAtStartPar
El proceso termina cuando el usuario ingresa 0. El número 0 no se incluye en los cálculos.

\sphinxAtStartPar
La función muestra por pantalla:
\sphinxhyphen{} El número más grande
\sphinxhyphen{} El número más pequeño
\sphinxhyphen{} La media de los números ingresados

\end{fulllineitems}



\section{Practica 5.10}
\label{\detokenize{pr5:module-pr5.10}}\label{\detokenize{pr5:practica-5-10}}\index{module@\spxentry{module}!pr5.10@\spxentry{pr5.10}}\index{pr5.10@\spxentry{pr5.10}!module@\spxentry{module}}\index{es\_numero\_perfecto() (en el módulo pr5.10)@\spxentry{es\_numero\_perfecto()}\spxextra{en el módulo pr5.10}}

\begin{fulllineitems}
\phantomsection\label{\detokenize{pr5:pr5.10.es_numero_perfecto}}
\pysigstartsignatures
\pysiglinewithargsret{\sphinxcode{\sphinxupquote{pr5.10.}}\sphinxbfcode{\sphinxupquote{es\_numero\_perfecto}}}{\sphinxparam{\DUrole{n}{n}}}{}
\pysigstopsignatures
\sphinxAtStartPar
Determina si un número entero dado es perfecto.

\sphinxAtStartPar
Un número perfecto es aquel que es igual a la suma de sus divisores propios
(excluyendo el mismo número). Por ejemplo, el número 6 es perfecto porque
sus divisores son 1, 2 y 3, y 1 + 2 + 3 = 6.

\sphinxAtStartPar
Parámetros:
n (int): El número a verificar.

\sphinxAtStartPar
Retorna:
bool: True si el número es perfecto, False si no lo es.

\end{fulllineitems}



\section{Practica 5.11}
\label{\detokenize{pr5:module-pr5.11}}\label{\detokenize{pr5:practica-5-11}}\index{module@\spxentry{module}!pr5.11@\spxentry{pr5.11}}\index{pr5.11@\spxentry{pr5.11}!module@\spxentry{module}}\index{calcular\_producto\_digitos() (en el módulo pr5.11)@\spxentry{calcular\_producto\_digitos()}\spxextra{en el módulo pr5.11}}

\begin{fulllineitems}
\phantomsection\label{\detokenize{pr5:pr5.11.calcular_producto_digitos}}
\pysigstartsignatures
\pysiglinewithargsret{\sphinxcode{\sphinxupquote{pr5.11.}}\sphinxbfcode{\sphinxupquote{calcular\_producto\_digitos}}}{\sphinxparam{\DUrole{n}{numero}\DUrole{p}{:}\DUrole{w}{ }\DUrole{n}{int}}}{{ $\rightarrow$ int}}
\pysigstopsignatures
\sphinxAtStartPar
Calcula el producto de los dígitos de un número entero.

\sphinxAtStartPar
Dado un número entero, la función calcula el producto de todos sus dígitos.
Por ejemplo, con el número 123, el producto sería 1 * 2 * 3 = 6.

\sphinxAtStartPar
Parámetros:
numero (int): El número entero cuyo producto de dígitos se va a calcular.

\sphinxAtStartPar
Retorna:
int: El producto de los dígitos del número.

\end{fulllineitems}

\index{obtener\_numero() (en el módulo pr5.11)@\spxentry{obtener\_numero()}\spxextra{en el módulo pr5.11}}

\begin{fulllineitems}
\phantomsection\label{\detokenize{pr5:pr5.11.obtener_numero}}
\pysigstartsignatures
\pysiglinewithargsret{\sphinxcode{\sphinxupquote{pr5.11.}}\sphinxbfcode{\sphinxupquote{obtener\_numero}}}{}{{ $\rightarrow$ int}}
\pysigstopsignatures
\sphinxAtStartPar
Solicita un número entero al usuario una vez.

\sphinxAtStartPar
Retorna:
int: El número entero ingresado por el usuario.

\end{fulllineitems}


\sphinxstepscope


\chapter{Practica 6}
\label{\detokenize{pr6:practica-6}}\label{\detokenize{pr6::doc}}

\section{Practica 6.1.1}
\label{\detokenize{pr6:module-pr6.1_1}}\label{\detokenize{pr6:practica-6-1-1}}\index{module@\spxentry{module}!pr6.1\_1@\spxentry{pr6.1\_1}}\index{pr6.1\_1@\spxentry{pr6.1\_1}!module@\spxentry{module}}\index{concatenar\_vectores() (en el módulo pr6.1\_1)@\spxentry{concatenar\_vectores()}\spxextra{en el módulo pr6.1\_1}}

\begin{fulllineitems}
\phantomsection\label{\detokenize{pr6:pr6.1_1.concatenar_vectores}}
\pysigstartsignatures
\pysiglinewithargsret{\sphinxcode{\sphinxupquote{pr6.1\_1.}}\sphinxbfcode{\sphinxupquote{concatenar\_vectores}}}{\sphinxparam{\DUrole{n}{N}}\sphinxparamcomma \sphinxparam{\DUrole{n}{M}}}{}
\pysigstopsignatures
\sphinxAtStartPar
Dada dos vectores de enteros, de tamaños N y M respectivamente, esta función concatena ambos vectores
en un nuevo vector que tiene un tamaño de N+M.

\sphinxAtStartPar
Argumentos:
N \textendash{} Primer vector de enteros (array de numpy o lista de Python).
M \textendash{} Segundo vector de enteros (array de numpy o lista de Python).

\sphinxAtStartPar
Retorna:
Un nuevo vector que es la concatenación de N y M.

\end{fulllineitems}



\section{Practica 6.1.2}
\label{\detokenize{pr6:module-pr6.1_2}}\label{\detokenize{pr6:practica-6-1-2}}\index{module@\spxentry{module}!pr6.1\_2@\spxentry{pr6.1\_2}}\index{pr6.1\_2@\spxentry{pr6.1\_2}!module@\spxentry{module}}\index{visualizar\_hasta\_valor() (en el módulo pr6.1\_2)@\spxentry{visualizar\_hasta\_valor()}\spxextra{en el módulo pr6.1\_2}}

\begin{fulllineitems}
\phantomsection\label{\detokenize{pr6:pr6.1_2.visualizar_hasta_valor}}
\pysigstartsignatures
\pysiglinewithargsret{\sphinxcode{\sphinxupquote{pr6.1\_2.}}\sphinxbfcode{\sphinxupquote{visualizar\_hasta\_valor}}}{\sphinxparam{\DUrole{n}{vector}}\sphinxparamcomma \sphinxparam{\DUrole{n}{A}}}{}
\pysigstopsignatures
\sphinxAtStartPar
Dado un vector de enteros de tamaño N, esta función visualiza los valores de los elementos
hasta encontrar el primer elemento cuyo valor sea mayor o igual a un número A inclusive.
El resto de los elementos no se visualizarán.

\sphinxAtStartPar
Argumentos:
vector \textendash{} El vector de enteros (array de numpy).
A \textendash{} El número límite para visualizar los elementos.

\end{fulllineitems}



\section{Practica 6.1.3}
\label{\detokenize{pr6:module-pr6.1_3}}\label{\detokenize{pr6:practica-6-1-3}}\index{module@\spxentry{module}!pr6.1\_3@\spxentry{pr6.1\_3}}\index{pr6.1\_3@\spxentry{pr6.1\_3}!module@\spxentry{module}}\index{productoEscalar() (en el módulo pr6.1\_3)@\spxentry{productoEscalar()}\spxextra{en el módulo pr6.1\_3}}

\begin{fulllineitems}
\phantomsection\label{\detokenize{pr6:pr6.1_3.productoEscalar}}
\pysigstartsignatures
\pysiglinewithargsret{\sphinxcode{\sphinxupquote{pr6.1\_3.}}\sphinxbfcode{\sphinxupquote{productoEscalar}}}{\sphinxparam{\DUrole{n}{vectorA}}\sphinxparamcomma \sphinxparam{\DUrole{n}{vectorB}}}{}
\pysigstopsignatures
\sphinxAtStartPar
Calcula el producto escalar de dos vectores de 3 componentes.

\sphinxAtStartPar
Argumentos:
vectorA \textendash{} Primer vector (numpy array).
vectorB \textendash{} Segundo vector (numpy array).

\sphinxAtStartPar
Retorna:
El resultado del producto escalar entre vectorA y vectorB.

\end{fulllineitems}

\index{resta() (en el módulo pr6.1\_3)@\spxentry{resta()}\spxextra{en el módulo pr6.1\_3}}

\begin{fulllineitems}
\phantomsection\label{\detokenize{pr6:pr6.1_3.resta}}
\pysigstartsignatures
\pysiglinewithargsret{\sphinxcode{\sphinxupquote{pr6.1\_3.}}\sphinxbfcode{\sphinxupquote{resta}}}{\sphinxparam{\DUrole{n}{vectorA}}\sphinxparamcomma \sphinxparam{\DUrole{n}{vectorB}}}{}
\pysigstopsignatures
\sphinxAtStartPar
Resta dos vectores de 3 componentes.

\sphinxAtStartPar
Argumentos:
vectorA \textendash{} Primer vector (numpy array).
vectorB \textendash{} Segundo vector (numpy array).

\sphinxAtStartPar
Retorna:
El vector resultado de la resta de vectorA y vectorB.

\end{fulllineitems}

\index{suma() (en el módulo pr6.1\_3)@\spxentry{suma()}\spxextra{en el módulo pr6.1\_3}}

\begin{fulllineitems}
\phantomsection\label{\detokenize{pr6:pr6.1_3.suma}}
\pysigstartsignatures
\pysiglinewithargsret{\sphinxcode{\sphinxupquote{pr6.1\_3.}}\sphinxbfcode{\sphinxupquote{suma}}}{\sphinxparam{\DUrole{n}{vectorA}}\sphinxparamcomma \sphinxparam{\DUrole{n}{vectorB}}}{}
\pysigstopsignatures
\sphinxAtStartPar
Suma dos vectores de 3 componentes.

\sphinxAtStartPar
Argumentos:
vectorA \textendash{} Primer vector (numpy array).
vectorB \textendash{} Segundo vector (numpy array).

\sphinxAtStartPar
Retorna:
El vector resultado de la suma de vectorA y vectorB.

\end{fulllineitems}



\section{Practica 6.1.4}
\label{\detokenize{pr6:module-pr6.1_4}}\label{\detokenize{pr6:practica-6-1-4}}\index{module@\spxentry{module}!pr6.1\_4@\spxentry{pr6.1\_4}}\index{pr6.1\_4@\spxentry{pr6.1\_4}!module@\spxentry{module}}\index{ejecutar\_calculos() (en el módulo pr6.1\_4)@\spxentry{ejecutar\_calculos()}\spxextra{en el módulo pr6.1\_4}}

\begin{fulllineitems}
\phantomsection\label{\detokenize{pr6:pr6.1_4.ejecutar_calculos}}
\pysigstartsignatures
\pysiglinewithargsret{\sphinxcode{\sphinxupquote{pr6.1\_4.}}\sphinxbfcode{\sphinxupquote{ejecutar\_calculos}}}{}{}
\pysigstopsignatures
\sphinxAtStartPar
Solicita al usuario un vector de números reales y calcula el valor máximo, mínimo y la media del vector.

\sphinxAtStartPar
El usuario ingresa un número entero N, que representa la cantidad de elementos del vector,
seguido de los valores reales que conforman dicho vector. Luego se calculan y muestran:
\sphinxhyphen{} El valor máximo del vector.
\sphinxhyphen{} El valor mínimo del vector.
\sphinxhyphen{} La media de los valores del vector.

\sphinxAtStartPar
No retorna ningún valor, solo imprime los resultados en consola.

\end{fulllineitems}

\index{maximo() (en el módulo pr6.1\_4)@\spxentry{maximo()}\spxextra{en el módulo pr6.1\_4}}

\begin{fulllineitems}
\phantomsection\label{\detokenize{pr6:pr6.1_4.maximo}}
\pysigstartsignatures
\pysiglinewithargsret{\sphinxcode{\sphinxupquote{pr6.1\_4.}}\sphinxbfcode{\sphinxupquote{maximo}}}{\sphinxparam{\DUrole{n}{vector}}}{}
\pysigstopsignatures
\sphinxAtStartPar
Devuelve el valor máximo de un vector de números reales.

\sphinxAtStartPar
Argumento:
vector \textendash{} Un array de números reales.

\sphinxAtStartPar
Retorna:
El valor máximo del vector.

\end{fulllineitems}

\index{media() (en el módulo pr6.1\_4)@\spxentry{media()}\spxextra{en el módulo pr6.1\_4}}

\begin{fulllineitems}
\phantomsection\label{\detokenize{pr6:pr6.1_4.media}}
\pysigstartsignatures
\pysiglinewithargsret{\sphinxcode{\sphinxupquote{pr6.1\_4.}}\sphinxbfcode{\sphinxupquote{media}}}{\sphinxparam{\DUrole{n}{vector}}}{}
\pysigstopsignatures
\sphinxAtStartPar
Devuelve la media de los valores de un vector de números reales.

\sphinxAtStartPar
Argumento:
vector \textendash{} Un array de números reales.

\sphinxAtStartPar
Retorna:
La media de los valores del vector.

\end{fulllineitems}

\index{minimo() (en el módulo pr6.1\_4)@\spxentry{minimo()}\spxextra{en el módulo pr6.1\_4}}

\begin{fulllineitems}
\phantomsection\label{\detokenize{pr6:pr6.1_4.minimo}}
\pysigstartsignatures
\pysiglinewithargsret{\sphinxcode{\sphinxupquote{pr6.1\_4.}}\sphinxbfcode{\sphinxupquote{minimo}}}{\sphinxparam{\DUrole{n}{vector}}}{}
\pysigstopsignatures
\sphinxAtStartPar
Devuelve el valor mínimo de un vector de números reales.

\sphinxAtStartPar
Argumento:
vector \textendash{} Un array de números reales.

\sphinxAtStartPar
Retorna:
El valor mínimo del vector.

\end{fulllineitems}



\section{Practica 6.1.5}
\label{\detokenize{pr6:module-pr6.1_5}}\label{\detokenize{pr6:practica-6-1-5}}\index{module@\spxentry{module}!pr6.1\_5@\spxentry{pr6.1\_5}}\index{pr6.1\_5@\spxentry{pr6.1\_5}!module@\spxentry{module}}\index{llenar\_y\_sumar\_matriz() (en el módulo pr6.1\_5)@\spxentry{llenar\_y\_sumar\_matriz()}\spxextra{en el módulo pr6.1\_5}}

\begin{fulllineitems}
\phantomsection\label{\detokenize{pr6:pr6.1_5.llenar_y_sumar_matriz}}
\pysigstartsignatures
\pysiglinewithargsret{\sphinxcode{\sphinxupquote{pr6.1\_5.}}\sphinxbfcode{\sphinxupquote{llenar\_y\_sumar\_matriz}}}{}{}
\pysigstopsignatures
\sphinxAtStartPar
Función que llena una matriz de 3x3 con valores aleatorios entre 0 y 20,
imprime la matriz, y luego imprime la suma de las filas y las columnas.

\end{fulllineitems}



\section{Practica 6.1.6}
\label{\detokenize{pr6:module-pr6.1_6}}\label{\detokenize{pr6:practica-6-1-6}}\index{module@\spxentry{module}!pr6.1\_6@\spxentry{pr6.1\_6}}\index{pr6.1\_6@\spxentry{pr6.1\_6}!module@\spxentry{module}}\index{operaciones\_matrices() (en el módulo pr6.1\_6)@\spxentry{operaciones\_matrices()}\spxextra{en el módulo pr6.1\_6}}

\begin{fulllineitems}
\phantomsection\label{\detokenize{pr6:pr6.1_6.operaciones_matrices}}
\pysigstartsignatures
\pysiglinewithargsret{\sphinxcode{\sphinxupquote{pr6.1\_6.}}\sphinxbfcode{\sphinxupquote{operaciones\_matrices}}}{}{}
\pysigstopsignatures
\sphinxAtStartPar
Función que realiza operaciones con dos matrices de 3x5:
\sphinxhyphen{} Suma
\sphinxhyphen{} Diferencia
\sphinxhyphen{} Producto elemento a elemento

\sphinxAtStartPar
Luego imprime los resultados de estas operaciones.

\end{fulllineitems}



\section{Practica 6.1.7}
\label{\detokenize{pr6:module-pr6.1_7}}\label{\detokenize{pr6:practica-6-1-7}}\index{module@\spxentry{module}!pr6.1\_7@\spxentry{pr6.1\_7}}\index{pr6.1\_7@\spxentry{pr6.1\_7}!module@\spxentry{module}}\index{resolver\_sistema\_ecuaciones() (en el módulo pr6.1\_7)@\spxentry{resolver\_sistema\_ecuaciones()}\spxextra{en el módulo pr6.1\_7}}

\begin{fulllineitems}
\phantomsection\label{\detokenize{pr6:pr6.1_7.resolver_sistema_ecuaciones}}
\pysigstartsignatures
\pysiglinewithargsret{\sphinxcode{\sphinxupquote{pr6.1\_7.}}\sphinxbfcode{\sphinxupquote{resolver\_sistema\_ecuaciones}}}{}{}
\pysigstopsignatures
\sphinxAtStartPar
Resuelve un sistema de tres ecuaciones lineales con tres incógnitas utilizando matrices.

\sphinxAtStartPar
El sistema de ecuaciones se representa como:
A * X = B
donde:
\begin{quote}

\sphinxAtStartPar
A es la matriz de coeficientes 3x3,
X es el vector de incógnitas 3x1,
B es el vector de términos independientes 3x1.
\end{quote}

\sphinxAtStartPar
El programa realiza los siguientes pasos:
1. Solicita al usuario ingresar los valores de la matriz de coeficientes (A) y del vector de términos independientes (B).
2. Calcula el determinante de la matriz de coeficientes A.
3. Si el determinante de A es cero, informa que el sistema no tiene solución única.
4. Si el determinante no es cero, calcula la matriz inversa de A.
5. Resuelve el sistema de ecuaciones utilizando la fórmula X = A⁻\(\sp{\text{1}}\) * B.
6. Muestra la solución del sistema, es decir, el vector de incógnitas X.

\sphinxAtStartPar
Ejemplo de uso:
El usuario debe ingresar las entradas de la matriz A y el vector B de la siguiente manera:
\sphinxhyphen{} Una matriz A de 3x3 con los coeficientes de las ecuaciones.
\sphinxhyphen{} Un vector B de 3x1 con los términos independientes.

\sphinxAtStartPar
El programa luego calcula y muestra el determinante de A, la matriz inversa de A (si es posible), y la solución del sistema de ecuaciones.

\sphinxAtStartPar
Requisitos:
\sphinxhyphen{} El sistema debe tener una solución única, lo que implica que el determinante de la matriz A debe ser diferente de cero.

\sphinxAtStartPar
Retorna:
None: La función imprime la solución en consola.

\end{fulllineitems}



\section{Practica 6.2.1}
\label{\detokenize{pr6:module-pr6.2_1}}\label{\detokenize{pr6:practica-6-2-1}}\index{module@\spxentry{module}!pr6.2\_1@\spxentry{pr6.2\_1}}\index{pr6.2\_1@\spxentry{pr6.2\_1}!module@\spxentry{module}}\index{contar\_vocales\_consonantes() (en el módulo pr6.2\_1)@\spxentry{contar\_vocales\_consonantes()}\spxextra{en el módulo pr6.2\_1}}

\begin{fulllineitems}
\phantomsection\label{\detokenize{pr6:pr6.2_1.contar_vocales_consonantes}}
\pysigstartsignatures
\pysiglinewithargsret{\sphinxcode{\sphinxupquote{pr6.2\_1.}}\sphinxbfcode{\sphinxupquote{contar\_vocales\_consonantes}}}{\sphinxparam{\DUrole{n}{palabra}}}{}
\pysigstopsignatures
\sphinxAtStartPar
Esta función toma una palabra y cuenta el número de vocales y consonantes que contiene.

\sphinxAtStartPar
Parámetros:
palabra (str): La palabra sobre la cual se va a contar las vocales y consonantes.

\sphinxAtStartPar
Retorna:
tuple: Un tuple con dos valores:
\begin{itemize}
\item {} 
\sphinxAtStartPar
El número de vocales en la palabra.

\item {} 
\sphinxAtStartPar
El número de consonantes en la palabra.

\end{itemize}

\sphinxAtStartPar
Ejemplo:
\textgreater{}\textgreater{}\textgreater{} contar\_vocales\_consonantes(«Hola»)
(2, 2)

\end{fulllineitems}



\section{Practica 6.2.2}
\label{\detokenize{pr6:module-pr6.2_2}}\label{\detokenize{pr6:practica-6-2-2}}\index{module@\spxentry{module}!pr6.2\_2@\spxentry{pr6.2\_2}}\index{pr6.2\_2@\spxentry{pr6.2\_2}!module@\spxentry{module}}\index{quitar\_espacios() (en el módulo pr6.2\_2)@\spxentry{quitar\_espacios()}\spxextra{en el módulo pr6.2\_2}}

\begin{fulllineitems}
\phantomsection\label{\detokenize{pr6:pr6.2_2.quitar_espacios}}
\pysigstartsignatures
\pysiglinewithargsret{\sphinxcode{\sphinxupquote{pr6.2\_2.}}\sphinxbfcode{\sphinxupquote{quitar\_espacios}}}{\sphinxparam{\DUrole{n}{cadena\_car}}}{}
\pysigstopsignatures
\sphinxAtStartPar
Esta función recibe una cadena de caracteres y elimina los espacios en blanco al principio y al final.

\sphinxAtStartPar
Parámetros:
cadena\_car (str): La cadena de caracteres a la cual se le eliminarán los espacios en blanco.

\sphinxAtStartPar
Retorna:
str: La cadena corregida sin espacios en blanco al principio y al final.

\sphinxAtStartPar
Ejemplo:
\textgreater{}\textgreater{}\textgreater{} quitar\_espacios(»   hola como estas   «)
“hola como estas”

\end{fulllineitems}



\section{Practica 6.2.3}
\label{\detokenize{pr6:module-pr6.2_3}}\label{\detokenize{pr6:practica-6-2-3}}\index{module@\spxentry{module}!pr6.2\_3@\spxentry{pr6.2\_3}}\index{pr6.2\_3@\spxentry{pr6.2\_3}!module@\spxentry{module}}\index{eliminar\_a() (en el módulo pr6.2\_3)@\spxentry{eliminar\_a()}\spxextra{en el módulo pr6.2\_3}}

\begin{fulllineitems}
\phantomsection\label{\detokenize{pr6:pr6.2_3.eliminar_a}}
\pysigstartsignatures
\pysiglinewithargsret{\sphinxcode{\sphinxupquote{pr6.2\_3.}}\sphinxbfcode{\sphinxupquote{eliminar\_a}}}{\sphinxparam{\DUrole{n}{cadena}}}{}
\pysigstopsignatures
\sphinxAtStartPar
Esta función elimina todas las ocurrencias del caracter “a” en una cadena de caracteres.

\sphinxAtStartPar
Parámetros:
cadena (str): La cadena de caracteres de la cual se eliminarán todas las “a”.

\sphinxAtStartPar
Retorna:
str: La cadena sin las ocurrencias del caracter “a”.

\sphinxAtStartPar
Ejemplo:
\textgreater{}\textgreater{}\textgreater{} eliminar\_a(«banana»)
“bn”

\end{fulllineitems}



\section{Practica 6.2.4}
\label{\detokenize{pr6:module-pr6.2_4}}\label{\detokenize{pr6:practica-6-2-4}}\index{module@\spxentry{module}!pr6.2\_4@\spxentry{pr6.2\_4}}\index{pr6.2\_4@\spxentry{pr6.2\_4}!module@\spxentry{module}}\index{es\_palindromo() (en el módulo pr6.2\_4)@\spxentry{es\_palindromo()}\spxextra{en el módulo pr6.2\_4}}

\begin{fulllineitems}
\phantomsection\label{\detokenize{pr6:pr6.2_4.es_palindromo}}
\pysigstartsignatures
\pysiglinewithargsret{\sphinxcode{\sphinxupquote{pr6.2\_4.}}\sphinxbfcode{\sphinxupquote{es\_palindromo}}}{\sphinxparam{\DUrole{n}{cadena}}}{}
\pysigstopsignatures
\sphinxAtStartPar
Esta función comprueba si una cadena es un palíndromo. Una palabra es un palíndromo si se lee
igual de izquierda a derecha que de derecha a izquierda.

\sphinxAtStartPar
Parámetros:
cadena (str): La cadena que se desea comprobar.

\sphinxAtStartPar
Retorna:
bool: Retorna True si la cadena es un palíndromo, False en caso contrario.

\sphinxAtStartPar
Ejemplo:
\textgreater{}\textgreater{}\textgreater{} es\_palindromo(«reconocer»)
True
\textgreater{}\textgreater{}\textgreater{} es\_palindromo(«hola»)
False

\end{fulllineitems}



\section{Practica 6.3.1}
\label{\detokenize{pr6:module-pr6.3_1}}\label{\detokenize{pr6:practica-6-3-1}}\index{module@\spxentry{module}!pr6.3\_1@\spxentry{pr6.3\_1}}\index{pr6.3\_1@\spxentry{pr6.3\_1}!module@\spxentry{module}}\index{lista\_metodos() (en el módulo pr6.3\_1)@\spxentry{lista\_metodos()}\spxextra{en el módulo pr6.3\_1}}

\begin{fulllineitems}
\phantomsection\label{\detokenize{pr6:pr6.3_1.lista_metodos}}
\pysigstartsignatures
\pysiglinewithargsret{\sphinxcode{\sphinxupquote{pr6.3\_1.}}\sphinxbfcode{\sphinxupquote{lista\_metodos}}}{}{}
\pysigstopsignatures
\sphinxAtStartPar
Esta función demuestra la utilidad de varios métodos de listas en Python.
\begin{enumerate}
\sphinxsetlistlabels{\arabic}{enumi}{enumii}{}{.}%
\item {} 
\sphinxAtStartPar
Ordena una lista de menor a mayor utilizando el método sort().

\item {} 
\sphinxAtStartPar
Ordena la misma lista de mayor a menor utilizando sort(reverse=True).

\item {} 
\sphinxAtStartPar
Elimina el último elemento de la lista utilizando el método pop().

\item {} 
\sphinxAtStartPar
Muestra la posición de un elemento en una lista utilizando index().

\item {} 
\sphinxAtStartPar
Añade un nuevo elemento a la lista utilizando append().

\item {} 
\sphinxAtStartPar
Elimina un elemento específico de la lista utilizando remove().

\end{enumerate}

\sphinxAtStartPar
Ejemplo:
\textgreater{}\textgreater{}\textgreater{} lista\_metodos()

\end{fulllineitems}


\sphinxstepscope


\chapter{Practica 7}
\label{\detokenize{pr7:practica-7}}\label{\detokenize{pr7::doc}}

\section{Practica 7.1}
\label{\detokenize{pr7:module-pr7.1}}\label{\detokenize{pr7:practica-7-1}}\index{module@\spxentry{module}!pr7.1@\spxentry{pr7.1}}\index{pr7.1@\spxentry{pr7.1}!module@\spxentry{module}}\index{grafico\_1() (en el módulo pr7.1)@\spxentry{grafico\_1()}\spxextra{en el módulo pr7.1}}

\begin{fulllineitems}
\phantomsection\label{\detokenize{pr7:pr7.1.grafico_1}}
\pysigstartsignatures
\pysiglinewithargsret{\sphinxcode{\sphinxupquote{pr7.1.}}\sphinxbfcode{\sphinxupquote{grafico\_1}}}{}{}
\pysigstopsignatures
\sphinxAtStartPar
Grafica la función y = 3x \sphinxhyphen{} x\textasciicircum{}3 en el intervalo {[}\sphinxhyphen{}10, 10{]}.

\end{fulllineitems}

\index{grafico\_10() (en el módulo pr7.1)@\spxentry{grafico\_10()}\spxextra{en el módulo pr7.1}}

\begin{fulllineitems}
\phantomsection\label{\detokenize{pr7:pr7.1.grafico_10}}
\pysigstartsignatures
\pysiglinewithargsret{\sphinxcode{\sphinxupquote{pr7.1.}}\sphinxbfcode{\sphinxupquote{grafico\_10}}}{}{}
\pysigstopsignatures
\sphinxAtStartPar
Grafica la función y = (x \sphinxhyphen{} 1) * exp(\sphinxhyphen{}x) en el intervalo {[}\sphinxhyphen{}30, 30{]}.

\end{fulllineitems}

\index{grafico\_11() (en el módulo pr7.1)@\spxentry{grafico\_11()}\spxextra{en el módulo pr7.1}}

\begin{fulllineitems}
\phantomsection\label{\detokenize{pr7:pr7.1.grafico_11}}
\pysigstartsignatures
\pysiglinewithargsret{\sphinxcode{\sphinxupquote{pr7.1.}}\sphinxbfcode{\sphinxupquote{grafico\_11}}}{}{}
\pysigstopsignatures
\sphinxAtStartPar
Grafica la función y = log(x) / x en el intervalo {[}1, 30{]}.

\end{fulllineitems}

\index{grafico\_12() (en el módulo pr7.1)@\spxentry{grafico\_12()}\spxextra{en el módulo pr7.1}}

\begin{fulllineitems}
\phantomsection\label{\detokenize{pr7:pr7.1.grafico_12}}
\pysigstartsignatures
\pysiglinewithargsret{\sphinxcode{\sphinxupquote{pr7.1.}}\sphinxbfcode{\sphinxupquote{grafico\_12}}}{}{}
\pysigstopsignatures
\sphinxAtStartPar
Grafica la función y(x) = sin(4πx) * e\textasciicircum{}(\sphinxhyphen{}5x) en el intervalo {[}0, 1{]}.

\end{fulllineitems}

\index{grafico\_13() (en el módulo pr7.1)@\spxentry{grafico\_13()}\spxextra{en el módulo pr7.1}}

\begin{fulllineitems}
\phantomsection\label{\detokenize{pr7:pr7.1.grafico_13}}
\pysigstartsignatures
\pysiglinewithargsret{\sphinxcode{\sphinxupquote{pr7.1.}}\sphinxbfcode{\sphinxupquote{grafico\_13}}}{}{}
\pysigstopsignatures
\sphinxAtStartPar
Grafica la función y(x) = cos(2πx) * e\textasciicircum{}(\sphinxhyphen{}x) en el intervalo {[}0, 5{]}.

\end{fulllineitems}

\index{grafico\_2() (en el módulo pr7.1)@\spxentry{grafico\_2()}\spxextra{en el módulo pr7.1}}

\begin{fulllineitems}
\phantomsection\label{\detokenize{pr7:pr7.1.grafico_2}}
\pysigstartsignatures
\pysiglinewithargsret{\sphinxcode{\sphinxupquote{pr7.1.}}\sphinxbfcode{\sphinxupquote{grafico\_2}}}{}{}
\pysigstopsignatures
\sphinxAtStartPar
Grafica la función y = x\textasciicircum{}4 \sphinxhyphen{} 2x\textasciicircum{}2 \sphinxhyphen{} 8 en el intervalo {[}\sphinxhyphen{}10, 10{]}.

\end{fulllineitems}

\index{grafico\_3() (en el módulo pr7.1)@\spxentry{grafico\_3()}\spxextra{en el módulo pr7.1}}

\begin{fulllineitems}
\phantomsection\label{\detokenize{pr7:pr7.1.grafico_3}}
\pysigstartsignatures
\pysiglinewithargsret{\sphinxcode{\sphinxupquote{pr7.1.}}\sphinxbfcode{\sphinxupquote{grafico\_3}}}{}{}
\pysigstopsignatures
\sphinxAtStartPar
Grafica la función y = (x\textasciicircum{}3) / (x \sphinxhyphen{} 1)\textasciicircum{}2 en el intervalo {[}\sphinxhyphen{}20, 20{]}.

\end{fulllineitems}

\index{grafico\_4() (en el módulo pr7.1)@\spxentry{grafico\_4()}\spxextra{en el módulo pr7.1}}

\begin{fulllineitems}
\phantomsection\label{\detokenize{pr7:pr7.1.grafico_4}}
\pysigstartsignatures
\pysiglinewithargsret{\sphinxcode{\sphinxupquote{pr7.1.}}\sphinxbfcode{\sphinxupquote{grafico\_4}}}{}{}
\pysigstopsignatures
\sphinxAtStartPar
Grafica la función y = (x\textasciicircum{}4 + 1) / x\textasciicircum{}2 en el intervalo {[}\sphinxhyphen{}20, 20{]}.

\end{fulllineitems}

\index{grafico\_5() (en el módulo pr7.1)@\spxentry{grafico\_5()}\spxextra{en el módulo pr7.1}}

\begin{fulllineitems}
\phantomsection\label{\detokenize{pr7:pr7.1.grafico_5}}
\pysigstartsignatures
\pysiglinewithargsret{\sphinxcode{\sphinxupquote{pr7.1.}}\sphinxbfcode{\sphinxupquote{grafico\_5}}}{}{}
\pysigstopsignatures
\sphinxAtStartPar
Grafica la función y = (x\textasciicircum{}2) / (2 \sphinxhyphen{} x) en el intervalo {[}\sphinxhyphen{}20, 20{]}.

\end{fulllineitems}

\index{grafico\_6() (en el módulo pr7.1)@\spxentry{grafico\_6()}\spxextra{en el módulo pr7.1}}

\begin{fulllineitems}
\phantomsection\label{\detokenize{pr7:pr7.1.grafico_6}}
\pysigstartsignatures
\pysiglinewithargsret{\sphinxcode{\sphinxupquote{pr7.1.}}\sphinxbfcode{\sphinxupquote{grafico\_6}}}{}{}
\pysigstopsignatures
\sphinxAtStartPar
Grafica la función y = x / (1 + x\textasciicircum{}2) en el intervalo {[}\sphinxhyphen{}20, 20{]}.

\end{fulllineitems}

\index{grafico\_7() (en el módulo pr7.1)@\spxentry{grafico\_7()}\spxextra{en el módulo pr7.1}}

\begin{fulllineitems}
\phantomsection\label{\detokenize{pr7:pr7.1.grafico_7}}
\pysigstartsignatures
\pysiglinewithargsret{\sphinxcode{\sphinxupquote{pr7.1.}}\sphinxbfcode{\sphinxupquote{grafico\_7}}}{}{}
\pysigstopsignatures
\sphinxAtStartPar
Grafica la función y = (x\textasciicircum{}2 \sphinxhyphen{} 3x + 2) / (x\textasciicircum{}2 + 1) en el intervalo {[}\sphinxhyphen{}20, 20{]}.

\end{fulllineitems}

\index{grafico\_8() (en el módulo pr7.1)@\spxentry{grafico\_8()}\spxextra{en el módulo pr7.1}}

\begin{fulllineitems}
\phantomsection\label{\detokenize{pr7:pr7.1.grafico_8}}
\pysigstartsignatures
\pysiglinewithargsret{\sphinxcode{\sphinxupquote{pr7.1.}}\sphinxbfcode{\sphinxupquote{grafico\_8}}}{}{}
\pysigstopsignatures
\sphinxAtStartPar
Grafica la función y = x + sqrt(x) en el intervalo {[}\sphinxhyphen{}30, 30{]}.

\end{fulllineitems}

\index{grafico\_9() (en el módulo pr7.1)@\spxentry{grafico\_9()}\spxextra{en el módulo pr7.1}}

\begin{fulllineitems}
\phantomsection\label{\detokenize{pr7:pr7.1.grafico_9}}
\pysigstartsignatures
\pysiglinewithargsret{\sphinxcode{\sphinxupquote{pr7.1.}}\sphinxbfcode{\sphinxupquote{grafico\_9}}}{}{}
\pysigstopsignatures
\sphinxAtStartPar
Grafica la función y = exp(1/x) en el intervalo {[}\sphinxhyphen{}20, 20{]}, sin incluir el valor 0.

\end{fulllineitems}



\section{Practica 7.2}
\label{\detokenize{pr7:module-pr7.2}}\label{\detokenize{pr7:practica-7-2}}\index{module@\spxentry{module}!pr7.2@\spxentry{pr7.2}}\index{pr7.2@\spxentry{pr7.2}!module@\spxentry{module}}\index{cuadrados\_y\_cubos() (en el módulo pr7.2)@\spxentry{cuadrados\_y\_cubos()}\spxextra{en el módulo pr7.2}}

\begin{fulllineitems}
\phantomsection\label{\detokenize{pr7:pr7.2.cuadrados_y_cubos}}
\pysigstartsignatures
\pysiglinewithargsret{\sphinxcode{\sphinxupquote{pr7.2.}}\sphinxbfcode{\sphinxupquote{cuadrados\_y\_cubos}}}{\sphinxparam{\DUrole{n}{tupla}}}{}
\pysigstopsignatures
\sphinxAtStartPar
Devuelve una tupla con los elementos elevados al cuadrado y al cubo.

\sphinxAtStartPar
Parámetros:
tupla (tuple): Tupla de números enteros.

\sphinxAtStartPar
Retorna:
tuple: Una tupla con dos sub\sphinxhyphen{}tuplas: una con los cuadrados y otra con los cubos de los elementos.

\end{fulllineitems}

\index{ejecutar\_funcion\_elegida() (en el módulo pr7.2)@\spxentry{ejecutar\_funcion\_elegida()}\spxextra{en el módulo pr7.2}}

\begin{fulllineitems}
\phantomsection\label{\detokenize{pr7:pr7.2.ejecutar_funcion_elegida}}
\pysigstartsignatures
\pysiglinewithargsret{\sphinxcode{\sphinxupquote{pr7.2.}}\sphinxbfcode{\sphinxupquote{ejecutar\_funcion\_elegida}}}{\sphinxparam{\DUrole{n}{opcion}}\sphinxparamcomma \sphinxparam{\DUrole{n}{tupla}}}{}
\pysigstopsignatures
\sphinxAtStartPar
Ejecuta la función elegida por el usuario, proporcionando resultados basados en la opción seleccionada.

\sphinxAtStartPar
Parámetros:
opcion (int): Número que representa la función que el usuario quiere ejecutar.
tupla (tuple): Tupla de números que se utilizará como parámetro para la función seleccionada.

\sphinxAtStartPar
Retorna:
None

\end{fulllineitems}

\index{invertir\_y\_longitud() (en el módulo pr7.2)@\spxentry{invertir\_y\_longitud()}\spxextra{en el módulo pr7.2}}

\begin{fulllineitems}
\phantomsection\label{\detokenize{pr7:pr7.2.invertir_y_longitud}}
\pysigstartsignatures
\pysiglinewithargsret{\sphinxcode{\sphinxupquote{pr7.2.}}\sphinxbfcode{\sphinxupquote{invertir\_y\_longitud}}}{\sphinxparam{\DUrole{n}{tupla}}}{}
\pysigstopsignatures
\sphinxAtStartPar
Devuelve una tupla invertida y su longitud.

\sphinxAtStartPar
Parámetros:
tupla (tuple): Tupla de números.

\sphinxAtStartPar
Retorna:
tuple: Una tupla con la versión invertida y la longitud de la tupla original.

\end{fulllineitems}

\index{min\_y\_max() (en el módulo pr7.2)@\spxentry{min\_y\_max()}\spxextra{en el módulo pr7.2}}

\begin{fulllineitems}
\phantomsection\label{\detokenize{pr7:pr7.2.min_y_max}}
\pysigstartsignatures
\pysiglinewithargsret{\sphinxcode{\sphinxupquote{pr7.2.}}\sphinxbfcode{\sphinxupquote{min\_y\_max}}}{\sphinxparam{\DUrole{n}{tupla}}}{}
\pysigstopsignatures
\sphinxAtStartPar
Devuelve el mínimo y el máximo de los elementos de la tupla.

\sphinxAtStartPar
Parámetros:
tupla (tuple): Tupla de números enteros.

\sphinxAtStartPar
Retorna:
tuple: Una tupla con el mínimo y el máximo de los elementos.

\end{fulllineitems}

\index{pares\_e\_impares() (en el módulo pr7.2)@\spxentry{pares\_e\_impares()}\spxextra{en el módulo pr7.2}}

\begin{fulllineitems}
\phantomsection\label{\detokenize{pr7:pr7.2.pares_e_impares}}
\pysigstartsignatures
\pysiglinewithargsret{\sphinxcode{\sphinxupquote{pr7.2.}}\sphinxbfcode{\sphinxupquote{pares\_e\_impares}}}{\sphinxparam{\DUrole{n}{tupla}}}{}
\pysigstopsignatures
\sphinxAtStartPar
Devuelve una tupla con los elementos pares e impares separados.

\sphinxAtStartPar
Parámetros:
tupla (tuple): Tupla de números enteros.

\sphinxAtStartPar
Retorna:
tuple: Una tupla con dos sub\sphinxhyphen{}tuplas: una con los números pares y otra con los números impares.

\end{fulllineitems}

\index{suma\_y\_producto() (en el módulo pr7.2)@\spxentry{suma\_y\_producto()}\spxextra{en el módulo pr7.2}}

\begin{fulllineitems}
\phantomsection\label{\detokenize{pr7:pr7.2.suma_y_producto}}
\pysigstartsignatures
\pysiglinewithargsret{\sphinxcode{\sphinxupquote{pr7.2.}}\sphinxbfcode{\sphinxupquote{suma\_y\_producto}}}{\sphinxparam{\DUrole{n}{tupla}}}{}
\pysigstopsignatures
\sphinxAtStartPar
Devuelve la suma y el producto de los elementos de la tupla.

\sphinxAtStartPar
Parámetros:
tupla (tuple): Tupla de números enteros.

\sphinxAtStartPar
Retorna:
tuple: Una tupla con la suma y el producto de los elementos.

\end{fulllineitems}



\section{Practica 7.3}
\label{\detokenize{pr7:module-pr7.3}}\label{\detokenize{pr7:practica-7-3}}\index{module@\spxentry{module}!pr7.3@\spxentry{pr7.3}}\index{pr7.3@\spxentry{pr7.3}!module@\spxentry{module}}
\sphinxAtStartPar
Un diccionario en Python es una estructura de datos que almacena pares de clave\sphinxhyphen{}valor.
Es mutable y permite un acceso rápido a los valores mediante las claves.
Se define utilizando llaves \{\} y los pares de clave\sphinxhyphen{}valor se separan por comas.

\sphinxAtStartPar
Métodos útiles de los diccionarios:
\begin{enumerate}
\sphinxsetlistlabels{\arabic}{enumi}{enumii}{}{.}%
\item {} 
\sphinxAtStartPar
get(): Devuelve el valor de una clave, o un valor por defecto si la clave no existe.

\item {} 
\sphinxAtStartPar
keys(): Devuelve una vista de las claves del diccionario.

\item {} 
\sphinxAtStartPar
values(): Devuelve una vista de los valores del diccionario.

\item {} 
\sphinxAtStartPar
items(): Devuelve una vista de los pares clave\sphinxhyphen{}valor del diccionario.

\item {} 
\sphinxAtStartPar
update(): Actualiza el diccionario con los pares clave\sphinxhyphen{}valor de otro diccionario.

\item {} 
\sphinxAtStartPar
pop(): Elimina una clave y devuelve su valor.

\end{enumerate}

\sphinxstepscope


\chapter{Practica 8}
\label{\detokenize{pr8:practica-8}}\label{\detokenize{pr8::doc}}

\section{Practica 8.1}
\label{\detokenize{pr8:module-pr8.1}}\label{\detokenize{pr8:practica-8-1}}\index{module@\spxentry{module}!pr8.1@\spxentry{pr8.1}}\index{pr8.1@\spxentry{pr8.1}!module@\spxentry{module}}\index{Calculadora (clase en pr8.1)@\spxentry{Calculadora}\spxextra{clase en pr8.1}}

\begin{fulllineitems}
\phantomsection\label{\detokenize{pr8:pr8.1.Calculadora}}
\pysigstartsignatures
\pysiglinewithargsret{\sphinxbfcode{\sphinxupquote{class\DUrole{w}{ }}}\sphinxcode{\sphinxupquote{pr8.1.}}\sphinxbfcode{\sphinxupquote{Calculadora}}}{\sphinxparam{\DUrole{n}{numero1}}\sphinxparamcomma \sphinxparam{\DUrole{n}{numero2}}}{}
\pysigstopsignatures
\sphinxAtStartPar
Bases: \sphinxcode{\sphinxupquote{object}}

\sphinxAtStartPar
La clase Calculadora permite realizar operaciones básicas entre dos números: suma, resta, multiplicación y división.
\begin{description}
\sphinxlineitem{Atributos:}
\sphinxAtStartPar
numero1 (float): El primer número para las operaciones.
numero2 (float): El segundo número para las operaciones.

\sphinxlineitem{Métodos:}
\sphinxAtStartPar
\_\_init\_\_(self, numero1, numero2): Inicializa los números para las operaciones.
sumar(self): Devuelve la suma de los dos números.
restar(self): Devuelve la resta entre los dos números.
multiplicar(self): Devuelve la multiplicación de los dos números.
dividir(self): Devuelve la división de los dos números, o un error si el segundo número es cero.

\end{description}
\index{dividir() (método de pr8.1.Calculadora)@\spxentry{dividir()}\spxextra{método de pr8.1.Calculadora}}

\begin{fulllineitems}
\phantomsection\label{\detokenize{pr8:pr8.1.Calculadora.dividir}}
\pysigstartsignatures
\pysiglinewithargsret{\sphinxbfcode{\sphinxupquote{dividir}}}{}{}
\pysigstopsignatures
\sphinxAtStartPar
Devuelve la división de los dos números.

\sphinxAtStartPar
Si el segundo número es 0, devuelve un mensaje de error.

\end{fulllineitems}

\index{multiplicar() (método de pr8.1.Calculadora)@\spxentry{multiplicar()}\spxextra{método de pr8.1.Calculadora}}

\begin{fulllineitems}
\phantomsection\label{\detokenize{pr8:pr8.1.Calculadora.multiplicar}}
\pysigstartsignatures
\pysiglinewithargsret{\sphinxbfcode{\sphinxupquote{multiplicar}}}{}{}
\pysigstopsignatures
\sphinxAtStartPar
Devuelve la multiplicación de los dos números.

\end{fulllineitems}

\index{restar() (método de pr8.1.Calculadora)@\spxentry{restar()}\spxextra{método de pr8.1.Calculadora}}

\begin{fulllineitems}
\phantomsection\label{\detokenize{pr8:pr8.1.Calculadora.restar}}
\pysigstartsignatures
\pysiglinewithargsret{\sphinxbfcode{\sphinxupquote{restar}}}{}{}
\pysigstopsignatures
\sphinxAtStartPar
Devuelve la resta entre los dos números.

\end{fulllineitems}

\index{sumar() (método de pr8.1.Calculadora)@\spxentry{sumar()}\spxextra{método de pr8.1.Calculadora}}

\begin{fulllineitems}
\phantomsection\label{\detokenize{pr8:pr8.1.Calculadora.sumar}}
\pysigstartsignatures
\pysiglinewithargsret{\sphinxbfcode{\sphinxupquote{sumar}}}{}{}
\pysigstopsignatures
\sphinxAtStartPar
Devuelve la suma de los dos números.

\end{fulllineitems}


\end{fulllineitems}



\section{Practica 8.2}
\label{\detokenize{pr8:module-pr8.2}}\label{\detokenize{pr8:practica-8-2}}\index{module@\spxentry{module}!pr8.2@\spxentry{pr8.2}}\index{pr8.2@\spxentry{pr8.2}!module@\spxentry{module}}\index{DataPlotter (clase en pr8.2)@\spxentry{DataPlotter}\spxextra{clase en pr8.2}}

\begin{fulllineitems}
\phantomsection\label{\detokenize{pr8:pr8.2.DataPlotter}}
\pysigstartsignatures
\pysiglinewithargsret{\sphinxbfcode{\sphinxupquote{class\DUrole{w}{ }}}\sphinxcode{\sphinxupquote{pr8.2.}}\sphinxbfcode{\sphinxupquote{DataPlotter}}}{\sphinxparam{\DUrole{n}{x}}\sphinxparamcomma \sphinxparam{\DUrole{n}{y}}}{}
\pysigstopsignatures
\sphinxAtStartPar
Bases: \sphinxcode{\sphinxupquote{object}}

\sphinxAtStartPar
La clase DataPlotter permite crear diferentes tipos de gráficos (línea, dispersión, barras) utilizando datos proporcionados.
\begin{description}
\sphinxlineitem{Atributos:}
\sphinxAtStartPar
x (list): Lista de valores para el eje X.
y (list): Lista de valores para el eje Y.
line\_color (str): Color de la línea en los gráficos (por defecto “blue”).
line\_width (float): Grosor de la línea (por defecto 2).
line\_style (str): Estilo de la línea (por defecto “\sphinxhyphen{}“).
marker (str): Tipo de marcador (por defecto None).

\end{description}
\index{bar() (método de pr8.2.DataPlotter)@\spxentry{bar()}\spxextra{método de pr8.2.DataPlotter}}

\begin{fulllineitems}
\phantomsection\label{\detokenize{pr8:pr8.2.DataPlotter.bar}}
\pysigstartsignatures
\pysiglinewithargsret{\sphinxbfcode{\sphinxupquote{bar}}}{\sphinxparam{\DUrole{n}{title}\DUrole{o}{=}\DUrole{default_value}{\textquotesingle{}Gráfico de Barras\textquotesingle{}}}\sphinxparamcomma \sphinxparam{\DUrole{n}{xlabel}\DUrole{o}{=}\DUrole{default_value}{\textquotesingle{}Eje X\textquotesingle{}}}\sphinxparamcomma \sphinxparam{\DUrole{n}{ylabel}\DUrole{o}{=}\DUrole{default_value}{\textquotesingle{}Eje Y\textquotesingle{}}}\sphinxparamcomma \sphinxparam{\DUrole{n}{grid}\DUrole{o}{=}\DUrole{default_value}{True}}}{}
\pysigstopsignatures
\sphinxAtStartPar
Crea un gráfico de barras con las propiedades configuradas.
\begin{quote}\begin{description}
\sphinxlineitem{Parámetros}\begin{itemize}
\item {} 
\sphinxAtStartPar
\sphinxstyleliteralstrong{\sphinxupquote{title}} \textendash{} Título de la gráfica (str)

\item {} 
\sphinxAtStartPar
\sphinxstyleliteralstrong{\sphinxupquote{xlabel}} \textendash{} Etiqueta del eje X (str)

\item {} 
\sphinxAtStartPar
\sphinxstyleliteralstrong{\sphinxupquote{ylabel}} \textendash{} Etiqueta del eje Y (str)

\item {} 
\sphinxAtStartPar
\sphinxstyleliteralstrong{\sphinxupquote{grid}} \textendash{} Muestra una cuadrícula si es True (bool)

\end{itemize}

\end{description}\end{quote}

\end{fulllineitems}

\index{plot() (método de pr8.2.DataPlotter)@\spxentry{plot()}\spxextra{método de pr8.2.DataPlotter}}

\begin{fulllineitems}
\phantomsection\label{\detokenize{pr8:pr8.2.DataPlotter.plot}}
\pysigstartsignatures
\pysiglinewithargsret{\sphinxbfcode{\sphinxupquote{plot}}}{\sphinxparam{\DUrole{n}{title}\DUrole{o}{=}\DUrole{default_value}{\textquotesingle{}Gráfica\textquotesingle{}}}\sphinxparamcomma \sphinxparam{\DUrole{n}{xlabel}\DUrole{o}{=}\DUrole{default_value}{\textquotesingle{}Eje X\textquotesingle{}}}\sphinxparamcomma \sphinxparam{\DUrole{n}{ylabel}\DUrole{o}{=}\DUrole{default_value}{\textquotesingle{}Eje Y\textquotesingle{}}}\sphinxparamcomma \sphinxparam{\DUrole{n}{grid}\DUrole{o}{=}\DUrole{default_value}{True}}}{}
\pysigstopsignatures
\sphinxAtStartPar
Crea una gráfica con las propiedades configuradas.
\begin{quote}\begin{description}
\sphinxlineitem{Parámetros}\begin{itemize}
\item {} 
\sphinxAtStartPar
\sphinxstyleliteralstrong{\sphinxupquote{title}} \textendash{} Título de la gráfica (str)

\item {} 
\sphinxAtStartPar
\sphinxstyleliteralstrong{\sphinxupquote{xlabel}} \textendash{} Etiqueta del eje X (str)

\item {} 
\sphinxAtStartPar
\sphinxstyleliteralstrong{\sphinxupquote{ylabel}} \textendash{} Etiqueta del eje Y (str)

\item {} 
\sphinxAtStartPar
\sphinxstyleliteralstrong{\sphinxupquote{grid}} \textendash{} Muestra una cuadrícula si es True (bool)

\end{itemize}

\end{description}\end{quote}

\end{fulllineitems}

\index{scatter() (método de pr8.2.DataPlotter)@\spxentry{scatter()}\spxextra{método de pr8.2.DataPlotter}}

\begin{fulllineitems}
\phantomsection\label{\detokenize{pr8:pr8.2.DataPlotter.scatter}}
\pysigstartsignatures
\pysiglinewithargsret{\sphinxbfcode{\sphinxupquote{scatter}}}{\sphinxparam{\DUrole{n}{title}\DUrole{o}{=}\DUrole{default_value}{\textquotesingle{}Diagrama de Dispersión\textquotesingle{}}}\sphinxparamcomma \sphinxparam{\DUrole{n}{xlabel}\DUrole{o}{=}\DUrole{default_value}{\textquotesingle{}Eje X\textquotesingle{}}}\sphinxparamcomma \sphinxparam{\DUrole{n}{ylabel}\DUrole{o}{=}\DUrole{default_value}{\textquotesingle{}Eje Y\textquotesingle{}}}\sphinxparamcomma \sphinxparam{\DUrole{n}{grid}\DUrole{o}{=}\DUrole{default_value}{True}}}{}
\pysigstopsignatures
\sphinxAtStartPar
Crea un diagrama de dispersión con las propiedades configuradas.
\begin{quote}\begin{description}
\sphinxlineitem{Parámetros}\begin{itemize}
\item {} 
\sphinxAtStartPar
\sphinxstyleliteralstrong{\sphinxupquote{title}} \textendash{} Título de la gráfica (str)

\item {} 
\sphinxAtStartPar
\sphinxstyleliteralstrong{\sphinxupquote{xlabel}} \textendash{} Etiqueta del eje X (str)

\item {} 
\sphinxAtStartPar
\sphinxstyleliteralstrong{\sphinxupquote{ylabel}} \textendash{} Etiqueta del eje Y (str)

\item {} 
\sphinxAtStartPar
\sphinxstyleliteralstrong{\sphinxupquote{grid}} \textendash{} Muestra una cuadrícula si es True (bool)

\end{itemize}

\end{description}\end{quote}

\end{fulllineitems}

\index{set\_properties() (método de pr8.2.DataPlotter)@\spxentry{set\_properties()}\spxextra{método de pr8.2.DataPlotter}}

\begin{fulllineitems}
\phantomsection\label{\detokenize{pr8:pr8.2.DataPlotter.set_properties}}
\pysigstartsignatures
\pysiglinewithargsret{\sphinxbfcode{\sphinxupquote{set\_properties}}}{\sphinxparam{\DUrole{n}{color}\DUrole{o}{=}\DUrole{default_value}{\textquotesingle{}blue\textquotesingle{}}}\sphinxparamcomma \sphinxparam{\DUrole{n}{width}\DUrole{o}{=}\DUrole{default_value}{2}}\sphinxparamcomma \sphinxparam{\DUrole{n}{style}\DUrole{o}{=}\DUrole{default_value}{\textquotesingle{}\sphinxhyphen{}\textquotesingle{}}}\sphinxparamcomma \sphinxparam{\DUrole{n}{marker}\DUrole{o}{=}\DUrole{default_value}{None}}}{}
\pysigstopsignatures
\sphinxAtStartPar
Configura las propiedades de la gráfica.
\begin{quote}\begin{description}
\sphinxlineitem{Parámetros}\begin{itemize}
\item {} 
\sphinxAtStartPar
\sphinxstyleliteralstrong{\sphinxupquote{color}} \textendash{} Color de la línea (str)

\item {} 
\sphinxAtStartPar
\sphinxstyleliteralstrong{\sphinxupquote{width}} \textendash{} Grosor de la línea (float)

\item {} 
\sphinxAtStartPar
\sphinxstyleliteralstrong{\sphinxupquote{style}} \textendash{} Estilo de la línea (str)

\item {} 
\sphinxAtStartPar
\sphinxstyleliteralstrong{\sphinxupquote{marker}} \textendash{} Tipo de marcador (str)

\end{itemize}

\end{description}\end{quote}

\end{fulllineitems}


\end{fulllineitems}



\section{Practica 8.3}
\label{\detokenize{pr8:module-pr8.3}}\label{\detokenize{pr8:practica-8-3}}\index{module@\spxentry{module}!pr8.3@\spxentry{pr8.3}}\index{pr8.3@\spxentry{pr8.3}!module@\spxentry{module}}\index{Cilindro (clase en pr8.3)@\spxentry{Cilindro}\spxextra{clase en pr8.3}}

\begin{fulllineitems}
\phantomsection\label{\detokenize{pr8:pr8.3.Cilindro}}
\pysigstartsignatures
\pysiglinewithargsret{\sphinxbfcode{\sphinxupquote{class\DUrole{w}{ }}}\sphinxcode{\sphinxupquote{pr8.3.}}\sphinxbfcode{\sphinxupquote{Cilindro}}}{\sphinxparam{\DUrole{n}{radio}}\sphinxparamcomma \sphinxparam{\DUrole{n}{altura}}}{}
\pysigstopsignatures
\sphinxAtStartPar
Bases: {\hyperref[\detokenize{pr8:pr8.3.Objeto3D}]{\sphinxcrossref{\sphinxcode{\sphinxupquote{Objeto3D}}}}}

\sphinxAtStartPar
Representa un cilindro, un objeto tridimensional.
\begin{description}
\sphinxlineitem{Atributos:}
\sphinxAtStartPar
radio (float): Radio de la base del cilindro.
altura (float): Altura del cilindro.

\end{description}
\index{area\_superficie() (método de pr8.3.Cilindro)@\spxentry{area\_superficie()}\spxextra{método de pr8.3.Cilindro}}

\begin{fulllineitems}
\phantomsection\label{\detokenize{pr8:pr8.3.Cilindro.area_superficie}}
\pysigstartsignatures
\pysiglinewithargsret{\sphinxbfcode{\sphinxupquote{area\_superficie}}}{}{}
\pysigstopsignatures
\sphinxAtStartPar
Calcula el área de superficie del cilindro.
\begin{quote}\begin{description}
\sphinxlineitem{Devuelve}
\sphinxAtStartPar
El área de superficie del cilindro.

\end{description}\end{quote}

\end{fulllineitems}

\index{volumen() (método de pr8.3.Cilindro)@\spxentry{volumen()}\spxextra{método de pr8.3.Cilindro}}

\begin{fulllineitems}
\phantomsection\label{\detokenize{pr8:pr8.3.Cilindro.volumen}}
\pysigstartsignatures
\pysiglinewithargsret{\sphinxbfcode{\sphinxupquote{volumen}}}{}{}
\pysigstopsignatures
\sphinxAtStartPar
Calcula el volumen del cilindro.
\begin{quote}\begin{description}
\sphinxlineitem{Devuelve}
\sphinxAtStartPar
El volumen del cilindro.

\end{description}\end{quote}

\end{fulllineitems}


\end{fulllineitems}

\index{Circulo (clase en pr8.3)@\spxentry{Circulo}\spxextra{clase en pr8.3}}

\begin{fulllineitems}
\phantomsection\label{\detokenize{pr8:pr8.3.Circulo}}
\pysigstartsignatures
\pysiglinewithargsret{\sphinxbfcode{\sphinxupquote{class\DUrole{w}{ }}}\sphinxcode{\sphinxupquote{pr8.3.}}\sphinxbfcode{\sphinxupquote{Circulo}}}{\sphinxparam{\DUrole{n}{radio}}}{}
\pysigstopsignatures
\sphinxAtStartPar
Bases: {\hyperref[\detokenize{pr8:pr8.3.Objeto2D}]{\sphinxcrossref{\sphinxcode{\sphinxupquote{Objeto2D}}}}}

\sphinxAtStartPar
Representa un círculo, un objeto bidimensional.
\begin{description}
\sphinxlineitem{Atributos:}
\sphinxAtStartPar
radio (float): Radio del círculo.

\end{description}
\index{area() (método de pr8.3.Circulo)@\spxentry{area()}\spxextra{método de pr8.3.Circulo}}

\begin{fulllineitems}
\phantomsection\label{\detokenize{pr8:pr8.3.Circulo.area}}
\pysigstartsignatures
\pysiglinewithargsret{\sphinxbfcode{\sphinxupquote{area}}}{}{}
\pysigstopsignatures
\sphinxAtStartPar
Calcula el área del círculo.
\begin{quote}\begin{description}
\sphinxlineitem{Devuelve}
\sphinxAtStartPar
El área del círculo.

\end{description}\end{quote}

\end{fulllineitems}

\index{perimetro() (método de pr8.3.Circulo)@\spxentry{perimetro()}\spxextra{método de pr8.3.Circulo}}

\begin{fulllineitems}
\phantomsection\label{\detokenize{pr8:pr8.3.Circulo.perimetro}}
\pysigstartsignatures
\pysiglinewithargsret{\sphinxbfcode{\sphinxupquote{perimetro}}}{}{}
\pysigstopsignatures
\sphinxAtStartPar
Calcula el perímetro del círculo.
\begin{quote}\begin{description}
\sphinxlineitem{Devuelve}
\sphinxAtStartPar
El perímetro del círculo.

\end{description}\end{quote}

\end{fulllineitems}


\end{fulllineitems}

\index{Cubo (clase en pr8.3)@\spxentry{Cubo}\spxextra{clase en pr8.3}}

\begin{fulllineitems}
\phantomsection\label{\detokenize{pr8:pr8.3.Cubo}}
\pysigstartsignatures
\pysiglinewithargsret{\sphinxbfcode{\sphinxupquote{class\DUrole{w}{ }}}\sphinxcode{\sphinxupquote{pr8.3.}}\sphinxbfcode{\sphinxupquote{Cubo}}}{\sphinxparam{\DUrole{n}{lado}}}{}
\pysigstopsignatures
\sphinxAtStartPar
Bases: {\hyperref[\detokenize{pr8:pr8.3.Objeto3D}]{\sphinxcrossref{\sphinxcode{\sphinxupquote{Objeto3D}}}}}

\sphinxAtStartPar
Representa un cubo, un objeto tridimensional.
\begin{description}
\sphinxlineitem{Atributos:}
\sphinxAtStartPar
lado (float): Longitud de un lado del cubo.

\end{description}
\index{area\_superficie() (método de pr8.3.Cubo)@\spxentry{area\_superficie()}\spxextra{método de pr8.3.Cubo}}

\begin{fulllineitems}
\phantomsection\label{\detokenize{pr8:pr8.3.Cubo.area_superficie}}
\pysigstartsignatures
\pysiglinewithargsret{\sphinxbfcode{\sphinxupquote{area\_superficie}}}{}{}
\pysigstopsignatures
\sphinxAtStartPar
Calcula el área de superficie del cubo.
\begin{quote}\begin{description}
\sphinxlineitem{Devuelve}
\sphinxAtStartPar
El área de superficie del cubo.

\end{description}\end{quote}

\end{fulllineitems}

\index{volumen() (método de pr8.3.Cubo)@\spxentry{volumen()}\spxextra{método de pr8.3.Cubo}}

\begin{fulllineitems}
\phantomsection\label{\detokenize{pr8:pr8.3.Cubo.volumen}}
\pysigstartsignatures
\pysiglinewithargsret{\sphinxbfcode{\sphinxupquote{volumen}}}{}{}
\pysigstopsignatures
\sphinxAtStartPar
Calcula el volumen del cubo.
\begin{quote}\begin{description}
\sphinxlineitem{Devuelve}
\sphinxAtStartPar
El volumen del cubo.

\end{description}\end{quote}

\end{fulllineitems}


\end{fulllineitems}

\index{Esfera (clase en pr8.3)@\spxentry{Esfera}\spxextra{clase en pr8.3}}

\begin{fulllineitems}
\phantomsection\label{\detokenize{pr8:pr8.3.Esfera}}
\pysigstartsignatures
\pysiglinewithargsret{\sphinxbfcode{\sphinxupquote{class\DUrole{w}{ }}}\sphinxcode{\sphinxupquote{pr8.3.}}\sphinxbfcode{\sphinxupquote{Esfera}}}{\sphinxparam{\DUrole{n}{radio}}}{}
\pysigstopsignatures
\sphinxAtStartPar
Bases: {\hyperref[\detokenize{pr8:pr8.3.Objeto3D}]{\sphinxcrossref{\sphinxcode{\sphinxupquote{Objeto3D}}}}}

\sphinxAtStartPar
Representa una esfera, un objeto tridimensional.
\begin{description}
\sphinxlineitem{Atributos:}
\sphinxAtStartPar
radio (float): Radio de la esfera.

\end{description}
\index{area\_superficie() (método de pr8.3.Esfera)@\spxentry{area\_superficie()}\spxextra{método de pr8.3.Esfera}}

\begin{fulllineitems}
\phantomsection\label{\detokenize{pr8:pr8.3.Esfera.area_superficie}}
\pysigstartsignatures
\pysiglinewithargsret{\sphinxbfcode{\sphinxupquote{area\_superficie}}}{}{}
\pysigstopsignatures
\sphinxAtStartPar
Calcula el área de superficie de la esfera.
\begin{quote}\begin{description}
\sphinxlineitem{Devuelve}
\sphinxAtStartPar
El área de superficie de la esfera.

\end{description}\end{quote}

\end{fulllineitems}

\index{volumen() (método de pr8.3.Esfera)@\spxentry{volumen()}\spxextra{método de pr8.3.Esfera}}

\begin{fulllineitems}
\phantomsection\label{\detokenize{pr8:pr8.3.Esfera.volumen}}
\pysigstartsignatures
\pysiglinewithargsret{\sphinxbfcode{\sphinxupquote{volumen}}}{}{}
\pysigstopsignatures
\sphinxAtStartPar
Calcula el volumen de la esfera.
\begin{quote}\begin{description}
\sphinxlineitem{Devuelve}
\sphinxAtStartPar
El volumen de la esfera.

\end{description}\end{quote}

\end{fulllineitems}


\end{fulllineitems}

\index{Objeto2D (clase en pr8.3)@\spxentry{Objeto2D}\spxextra{clase en pr8.3}}

\begin{fulllineitems}
\phantomsection\label{\detokenize{pr8:pr8.3.Objeto2D}}
\pysigstartsignatures
\pysigline{\sphinxbfcode{\sphinxupquote{class\DUrole{w}{ }}}\sphinxcode{\sphinxupquote{pr8.3.}}\sphinxbfcode{\sphinxupquote{Objeto2D}}}
\pysigstopsignatures
\sphinxAtStartPar
Bases: {\hyperref[\detokenize{pr8:pr8.3.ObjetoGeometrico}]{\sphinxcrossref{\sphinxcode{\sphinxupquote{ObjetoGeometrico}}}}}

\sphinxAtStartPar
Clase base para objetos geométricos bidimensionales (2D).
Hereda de la clase ObjetoGeometrico y debe implementar los métodos
para calcular el perímetro y el área en las subclases.
\index{area() (método de pr8.3.Objeto2D)@\spxentry{area()}\spxextra{método de pr8.3.Objeto2D}}

\begin{fulllineitems}
\phantomsection\label{\detokenize{pr8:pr8.3.Objeto2D.area}}
\pysigstartsignatures
\pysiglinewithargsret{\sphinxbfcode{\sphinxupquote{area}}}{}{}
\pysigstopsignatures
\sphinxAtStartPar
Calcula el área del objeto. Este método debe ser implementado por la subclase.
\begin{quote}\begin{description}
\sphinxlineitem{Muestra}
\sphinxAtStartPar
\sphinxstyleliteralstrong{\sphinxupquote{NotImplementedError}} \textendash{} Si no está implementado en la subclase.

\end{description}\end{quote}

\end{fulllineitems}

\index{perimetro() (método de pr8.3.Objeto2D)@\spxentry{perimetro()}\spxextra{método de pr8.3.Objeto2D}}

\begin{fulllineitems}
\phantomsection\label{\detokenize{pr8:pr8.3.Objeto2D.perimetro}}
\pysigstartsignatures
\pysiglinewithargsret{\sphinxbfcode{\sphinxupquote{perimetro}}}{}{}
\pysigstopsignatures
\sphinxAtStartPar
Calcula el perímetro del objeto. Este método debe ser implementado por la subclase.
\begin{quote}\begin{description}
\sphinxlineitem{Muestra}
\sphinxAtStartPar
\sphinxstyleliteralstrong{\sphinxupquote{NotImplementedError}} \textendash{} Si no está implementado en la subclase.

\end{description}\end{quote}

\end{fulllineitems}


\end{fulllineitems}

\index{Objeto3D (clase en pr8.3)@\spxentry{Objeto3D}\spxextra{clase en pr8.3}}

\begin{fulllineitems}
\phantomsection\label{\detokenize{pr8:pr8.3.Objeto3D}}
\pysigstartsignatures
\pysigline{\sphinxbfcode{\sphinxupquote{class\DUrole{w}{ }}}\sphinxcode{\sphinxupquote{pr8.3.}}\sphinxbfcode{\sphinxupquote{Objeto3D}}}
\pysigstopsignatures
\sphinxAtStartPar
Bases: {\hyperref[\detokenize{pr8:pr8.3.ObjetoGeometrico}]{\sphinxcrossref{\sphinxcode{\sphinxupquote{ObjetoGeometrico}}}}}

\sphinxAtStartPar
Clase base para objetos geométricos tridimensionales (3D).
Hereda de la clase ObjetoGeometrico y debe implementar los métodos
para calcular el volumen y el área de superficie en las subclases.
\index{area\_superficie() (método de pr8.3.Objeto3D)@\spxentry{area\_superficie()}\spxextra{método de pr8.3.Objeto3D}}

\begin{fulllineitems}
\phantomsection\label{\detokenize{pr8:pr8.3.Objeto3D.area_superficie}}
\pysigstartsignatures
\pysiglinewithargsret{\sphinxbfcode{\sphinxupquote{area\_superficie}}}{}{}
\pysigstopsignatures
\sphinxAtStartPar
Calcula el área de superficie del objeto. Este método debe ser implementado por la subclase.
\begin{quote}\begin{description}
\sphinxlineitem{Muestra}
\sphinxAtStartPar
\sphinxstyleliteralstrong{\sphinxupquote{NotImplementedError}} \textendash{} Si no está implementado en la subclase.

\end{description}\end{quote}

\end{fulllineitems}

\index{volumen() (método de pr8.3.Objeto3D)@\spxentry{volumen()}\spxextra{método de pr8.3.Objeto3D}}

\begin{fulllineitems}
\phantomsection\label{\detokenize{pr8:pr8.3.Objeto3D.volumen}}
\pysigstartsignatures
\pysiglinewithargsret{\sphinxbfcode{\sphinxupquote{volumen}}}{}{}
\pysigstopsignatures
\sphinxAtStartPar
Calcula el volumen del objeto. Este método debe ser implementado por la subclase.
\begin{quote}\begin{description}
\sphinxlineitem{Muestra}
\sphinxAtStartPar
\sphinxstyleliteralstrong{\sphinxupquote{NotImplementedError}} \textendash{} Si no está implementado en la subclase.

\end{description}\end{quote}

\end{fulllineitems}


\end{fulllineitems}

\index{ObjetoGeometrico (clase en pr8.3)@\spxentry{ObjetoGeometrico}\spxextra{clase en pr8.3}}

\begin{fulllineitems}
\phantomsection\label{\detokenize{pr8:pr8.3.ObjetoGeometrico}}
\pysigstartsignatures
\pysigline{\sphinxbfcode{\sphinxupquote{class\DUrole{w}{ }}}\sphinxcode{\sphinxupquote{pr8.3.}}\sphinxbfcode{\sphinxupquote{ObjetoGeometrico}}}
\pysigstopsignatures
\sphinxAtStartPar
Bases: \sphinxcode{\sphinxupquote{object}}

\sphinxAtStartPar
Clase base para todos los objetos geométricos.
Se usa como clase base para los objetos 2D y 3D.

\end{fulllineitems}

\index{Rectangulo (clase en pr8.3)@\spxentry{Rectangulo}\spxextra{clase en pr8.3}}

\begin{fulllineitems}
\phantomsection\label{\detokenize{pr8:pr8.3.Rectangulo}}
\pysigstartsignatures
\pysiglinewithargsret{\sphinxbfcode{\sphinxupquote{class\DUrole{w}{ }}}\sphinxcode{\sphinxupquote{pr8.3.}}\sphinxbfcode{\sphinxupquote{Rectangulo}}}{\sphinxparam{\DUrole{n}{base}}\sphinxparamcomma \sphinxparam{\DUrole{n}{altura}}}{}
\pysigstopsignatures
\sphinxAtStartPar
Bases: {\hyperref[\detokenize{pr8:pr8.3.Objeto2D}]{\sphinxcrossref{\sphinxcode{\sphinxupquote{Objeto2D}}}}}

\sphinxAtStartPar
Representa un rectángulo, un objeto bidimensional.
\begin{description}
\sphinxlineitem{Atributos:}
\sphinxAtStartPar
base (float): Base del rectángulo.
altura (float): Altura del rectángulo.

\end{description}
\index{area() (método de pr8.3.Rectangulo)@\spxentry{area()}\spxextra{método de pr8.3.Rectangulo}}

\begin{fulllineitems}
\phantomsection\label{\detokenize{pr8:pr8.3.Rectangulo.area}}
\pysigstartsignatures
\pysiglinewithargsret{\sphinxbfcode{\sphinxupquote{area}}}{}{}
\pysigstopsignatures
\sphinxAtStartPar
Calcula el área del rectángulo.
\begin{quote}\begin{description}
\sphinxlineitem{Devuelve}
\sphinxAtStartPar
El área del rectángulo.

\end{description}\end{quote}

\end{fulllineitems}

\index{perimetro() (método de pr8.3.Rectangulo)@\spxentry{perimetro()}\spxextra{método de pr8.3.Rectangulo}}

\begin{fulllineitems}
\phantomsection\label{\detokenize{pr8:pr8.3.Rectangulo.perimetro}}
\pysigstartsignatures
\pysiglinewithargsret{\sphinxbfcode{\sphinxupquote{perimetro}}}{}{}
\pysigstopsignatures
\sphinxAtStartPar
Calcula el perímetro del rectángulo.
\begin{quote}\begin{description}
\sphinxlineitem{Devuelve}
\sphinxAtStartPar
El perímetro del rectángulo.

\end{description}\end{quote}

\end{fulllineitems}


\end{fulllineitems}

\index{Triangulo (clase en pr8.3)@\spxentry{Triangulo}\spxextra{clase en pr8.3}}

\begin{fulllineitems}
\phantomsection\label{\detokenize{pr8:pr8.3.Triangulo}}
\pysigstartsignatures
\pysiglinewithargsret{\sphinxbfcode{\sphinxupquote{class\DUrole{w}{ }}}\sphinxcode{\sphinxupquote{pr8.3.}}\sphinxbfcode{\sphinxupquote{Triangulo}}}{\sphinxparam{\DUrole{n}{lado1}}\sphinxparamcomma \sphinxparam{\DUrole{n}{lado2}}\sphinxparamcomma \sphinxparam{\DUrole{n}{lado3}}\sphinxparamcomma \sphinxparam{\DUrole{n}{altura\_base}}\sphinxparamcomma \sphinxparam{\DUrole{n}{base}}}{}
\pysigstopsignatures
\sphinxAtStartPar
Bases: {\hyperref[\detokenize{pr8:pr8.3.Objeto2D}]{\sphinxcrossref{\sphinxcode{\sphinxupquote{Objeto2D}}}}}

\sphinxAtStartPar
Representa un triángulo, un objeto bidimensional.
\begin{description}
\sphinxlineitem{Atributos:}
\sphinxAtStartPar
lado1 (float): Longitud del primer lado.
lado2 (float): Longitud del segundo lado.
lado3 (float): Longitud del tercer lado.
altura\_base (float): Altura correspondiente a la base.
base (float): Longitud de la base.

\end{description}
\index{area() (método de pr8.3.Triangulo)@\spxentry{area()}\spxextra{método de pr8.3.Triangulo}}

\begin{fulllineitems}
\phantomsection\label{\detokenize{pr8:pr8.3.Triangulo.area}}
\pysigstartsignatures
\pysiglinewithargsret{\sphinxbfcode{\sphinxupquote{area}}}{}{}
\pysigstopsignatures
\sphinxAtStartPar
Calcula el área del triángulo.
\begin{quote}\begin{description}
\sphinxlineitem{Devuelve}
\sphinxAtStartPar
El área del triángulo.

\end{description}\end{quote}

\end{fulllineitems}

\index{perimetro() (método de pr8.3.Triangulo)@\spxentry{perimetro()}\spxextra{método de pr8.3.Triangulo}}

\begin{fulllineitems}
\phantomsection\label{\detokenize{pr8:pr8.3.Triangulo.perimetro}}
\pysigstartsignatures
\pysiglinewithargsret{\sphinxbfcode{\sphinxupquote{perimetro}}}{}{}
\pysigstopsignatures
\sphinxAtStartPar
Calcula el perímetro del triángulo.
\begin{quote}\begin{description}
\sphinxlineitem{Devuelve}
\sphinxAtStartPar
El perímetro del triángulo.

\end{description}\end{quote}

\end{fulllineitems}


\end{fulllineitems}



\section{Practica 8.4}
\label{\detokenize{pr8:module-pr8.4}}\label{\detokenize{pr8:practica-8-4}}\index{module@\spxentry{module}!pr8.4@\spxentry{pr8.4}}\index{pr8.4@\spxentry{pr8.4}!module@\spxentry{module}}\index{Automovil (clase en pr8.4)@\spxentry{Automovil}\spxextra{clase en pr8.4}}

\begin{fulllineitems}
\phantomsection\label{\detokenize{pr8:pr8.4.Automovil}}
\pysigstartsignatures
\pysiglinewithargsret{\sphinxbfcode{\sphinxupquote{class\DUrole{w}{ }}}\sphinxcode{\sphinxupquote{pr8.4.}}\sphinxbfcode{\sphinxupquote{Automovil}}}{\sphinxparam{\DUrole{n}{marca}}\sphinxparamcomma \sphinxparam{\DUrole{n}{modelo}}\sphinxparamcomma \sphinxparam{\DUrole{n}{consumo\_combustible}}\sphinxparamcomma \sphinxparam{\DUrole{n}{capacidad\_personas}}}{}
\pysigstopsignatures
\sphinxAtStartPar
Bases: {\hyperref[\detokenize{pr8:pr8.4.Vehiculo}]{\sphinxcrossref{\sphinxcode{\sphinxupquote{Vehiculo}}}}}

\sphinxAtStartPar
Representa un automóvil, un tipo de vehículo.
\begin{description}
\sphinxlineitem{Atributos:}
\sphinxAtStartPar
capacidad\_personas (int): Número de personas que el automóvil puede transportar.

\end{description}
\index{caracteristicas() (método de pr8.4.Automovil)@\spxentry{caracteristicas()}\spxextra{método de pr8.4.Automovil}}

\begin{fulllineitems}
\phantomsection\label{\detokenize{pr8:pr8.4.Automovil.caracteristicas}}
\pysigstartsignatures
\pysiglinewithargsret{\sphinxbfcode{\sphinxupquote{caracteristicas}}}{}{}
\pysigstopsignatures
\sphinxAtStartPar
Devuelve las características del automóvil como un string, incluyendo la capacidad de personas.
\begin{quote}\begin{description}
\sphinxlineitem{Devuelve}
\sphinxAtStartPar
Características del automóvil.

\end{description}\end{quote}

\end{fulllineitems}


\end{fulllineitems}

\index{Camion (clase en pr8.4)@\spxentry{Camion}\spxextra{clase en pr8.4}}

\begin{fulllineitems}
\phantomsection\label{\detokenize{pr8:pr8.4.Camion}}
\pysigstartsignatures
\pysiglinewithargsret{\sphinxbfcode{\sphinxupquote{class\DUrole{w}{ }}}\sphinxcode{\sphinxupquote{pr8.4.}}\sphinxbfcode{\sphinxupquote{Camion}}}{\sphinxparam{\DUrole{n}{marca}}\sphinxparamcomma \sphinxparam{\DUrole{n}{modelo}}\sphinxparamcomma \sphinxparam{\DUrole{n}{consumo\_combustible}}\sphinxparamcomma \sphinxparam{\DUrole{n}{capacidad\_carga}}}{}
\pysigstopsignatures
\sphinxAtStartPar
Bases: {\hyperref[\detokenize{pr8:pr8.4.Vehiculo}]{\sphinxcrossref{\sphinxcode{\sphinxupquote{Vehiculo}}}}}

\sphinxAtStartPar
Representa un camión, un tipo de vehículo con capacidad de carga.
\begin{description}
\sphinxlineitem{Atributos:}
\sphinxAtStartPar
capacidad\_carga (float): Capacidad de carga en toneladas.

\end{description}
\index{caracteristicas() (método de pr8.4.Camion)@\spxentry{caracteristicas()}\spxextra{método de pr8.4.Camion}}

\begin{fulllineitems}
\phantomsection\label{\detokenize{pr8:pr8.4.Camion.caracteristicas}}
\pysigstartsignatures
\pysiglinewithargsret{\sphinxbfcode{\sphinxupquote{caracteristicas}}}{}{}
\pysigstopsignatures
\sphinxAtStartPar
Devuelve las características del camión como un string, incluyendo la capacidad de carga.
\begin{quote}\begin{description}
\sphinxlineitem{Devuelve}
\sphinxAtStartPar
Características del camión.

\end{description}\end{quote}

\end{fulllineitems}

\index{puede\_transportar() (método de pr8.4.Camion)@\spxentry{puede\_transportar()}\spxextra{método de pr8.4.Camion}}

\begin{fulllineitems}
\phantomsection\label{\detokenize{pr8:pr8.4.Camion.puede_transportar}}
\pysigstartsignatures
\pysiglinewithargsret{\sphinxbfcode{\sphinxupquote{puede\_transportar}}}{\sphinxparam{\DUrole{n}{carga}}}{}
\pysigstopsignatures
\sphinxAtStartPar
Determina si el camión puede transportar una carga dada.
\begin{quote}\begin{description}
\sphinxlineitem{Parámetros}
\sphinxAtStartPar
\sphinxstyleliteralstrong{\sphinxupquote{carga}} \textendash{} Carga que se desea transportar en toneladas.

\sphinxlineitem{Devuelve}
\sphinxAtStartPar
True si el camión puede transportar la carga, False en caso contrario.

\end{description}\end{quote}

\end{fulllineitems}


\end{fulllineitems}

\index{Motocicleta (clase en pr8.4)@\spxentry{Motocicleta}\spxextra{clase en pr8.4}}

\begin{fulllineitems}
\phantomsection\label{\detokenize{pr8:pr8.4.Motocicleta}}
\pysigstartsignatures
\pysiglinewithargsret{\sphinxbfcode{\sphinxupquote{class\DUrole{w}{ }}}\sphinxcode{\sphinxupquote{pr8.4.}}\sphinxbfcode{\sphinxupquote{Motocicleta}}}{\sphinxparam{\DUrole{n}{marca}}\sphinxparamcomma \sphinxparam{\DUrole{n}{modelo}}\sphinxparamcomma \sphinxparam{\DUrole{n}{consumo\_combustible}}\sphinxparamcomma \sphinxparam{\DUrole{n}{tipo}}}{}
\pysigstopsignatures
\sphinxAtStartPar
Bases: {\hyperref[\detokenize{pr8:pr8.4.Vehiculo}]{\sphinxcrossref{\sphinxcode{\sphinxupquote{Vehiculo}}}}}

\sphinxAtStartPar
Representa una motocicleta, un tipo de vehículo.
\begin{description}
\sphinxlineitem{Atributos:}
\sphinxAtStartPar
tipo (str): Tipo de motocicleta (por ejemplo, «Deportiva», «Scooter»).

\end{description}
\index{caracteristicas() (método de pr8.4.Motocicleta)@\spxentry{caracteristicas()}\spxextra{método de pr8.4.Motocicleta}}

\begin{fulllineitems}
\phantomsection\label{\detokenize{pr8:pr8.4.Motocicleta.caracteristicas}}
\pysigstartsignatures
\pysiglinewithargsret{\sphinxbfcode{\sphinxupquote{caracteristicas}}}{}{}
\pysigstopsignatures
\sphinxAtStartPar
Devuelve las características de la motocicleta como un string, incluyendo el tipo de motocicleta.
\begin{quote}\begin{description}
\sphinxlineitem{Devuelve}
\sphinxAtStartPar
Características de la motocicleta.

\end{description}\end{quote}

\end{fulllineitems}


\end{fulllineitems}

\index{Objeto2D (clase en pr8.4)@\spxentry{Objeto2D}\spxextra{clase en pr8.4}}

\begin{fulllineitems}
\phantomsection\label{\detokenize{pr8:pr8.4.Objeto2D}}
\pysigstartsignatures
\pysigline{\sphinxbfcode{\sphinxupquote{class\DUrole{w}{ }}}\sphinxcode{\sphinxupquote{pr8.4.}}\sphinxbfcode{\sphinxupquote{Objeto2D}}}
\pysigstopsignatures
\sphinxAtStartPar
Bases: {\hyperref[\detokenize{pr8:pr8.4.ObjetoGeometrico}]{\sphinxcrossref{\sphinxcode{\sphinxupquote{ObjetoGeometrico}}}}}

\sphinxAtStartPar
Clase base para objetos geométricos bidimensionales (2D).
Hereda de la clase ObjetoGeometrico y debe implementar los métodos
para calcular el perímetro y el área en las subclases.
\index{area() (método de pr8.4.Objeto2D)@\spxentry{area()}\spxextra{método de pr8.4.Objeto2D}}

\begin{fulllineitems}
\phantomsection\label{\detokenize{pr8:pr8.4.Objeto2D.area}}
\pysigstartsignatures
\pysiglinewithargsret{\sphinxbfcode{\sphinxupquote{area}}}{}{}
\pysigstopsignatures
\sphinxAtStartPar
Calcula el área del objeto. Este método debe ser implementado por la subclase.
\begin{quote}\begin{description}
\sphinxlineitem{Muestra}
\sphinxAtStartPar
\sphinxstyleliteralstrong{\sphinxupquote{NotImplementedError}} \textendash{} Si no está implementado en la subclase.

\end{description}\end{quote}

\end{fulllineitems}

\index{perimetro() (método de pr8.4.Objeto2D)@\spxentry{perimetro()}\spxextra{método de pr8.4.Objeto2D}}

\begin{fulllineitems}
\phantomsection\label{\detokenize{pr8:pr8.4.Objeto2D.perimetro}}
\pysigstartsignatures
\pysiglinewithargsret{\sphinxbfcode{\sphinxupquote{perimetro}}}{}{}
\pysigstopsignatures
\sphinxAtStartPar
Calcula el perímetro del objeto. Este método debe ser implementado por la subclase.
\begin{quote}\begin{description}
\sphinxlineitem{Muestra}
\sphinxAtStartPar
\sphinxstyleliteralstrong{\sphinxupquote{NotImplementedError}} \textendash{} Si no está implementado en la subclase.

\end{description}\end{quote}

\end{fulllineitems}


\end{fulllineitems}

\index{Objeto3D (clase en pr8.4)@\spxentry{Objeto3D}\spxextra{clase en pr8.4}}

\begin{fulllineitems}
\phantomsection\label{\detokenize{pr8:pr8.4.Objeto3D}}
\pysigstartsignatures
\pysigline{\sphinxbfcode{\sphinxupquote{class\DUrole{w}{ }}}\sphinxcode{\sphinxupquote{pr8.4.}}\sphinxbfcode{\sphinxupquote{Objeto3D}}}
\pysigstopsignatures
\sphinxAtStartPar
Bases: {\hyperref[\detokenize{pr8:pr8.4.ObjetoGeometrico}]{\sphinxcrossref{\sphinxcode{\sphinxupquote{ObjetoGeometrico}}}}}

\sphinxAtStartPar
Clase base para objetos geométricos tridimensionales (3D).
Hereda de la clase ObjetoGeometrico y debe implementar los métodos
para calcular el volumen y el área de superficie en las subclases.
\index{area\_superficie() (método de pr8.4.Objeto3D)@\spxentry{area\_superficie()}\spxextra{método de pr8.4.Objeto3D}}

\begin{fulllineitems}
\phantomsection\label{\detokenize{pr8:pr8.4.Objeto3D.area_superficie}}
\pysigstartsignatures
\pysiglinewithargsret{\sphinxbfcode{\sphinxupquote{area\_superficie}}}{}{}
\pysigstopsignatures
\sphinxAtStartPar
Calcula el área de superficie del objeto. Este método debe ser implementado por la subclase.
\begin{quote}\begin{description}
\sphinxlineitem{Muestra}
\sphinxAtStartPar
\sphinxstyleliteralstrong{\sphinxupquote{NotImplementedError}} \textendash{} Si no está implementado en la subclase.

\end{description}\end{quote}

\end{fulllineitems}

\index{volumen() (método de pr8.4.Objeto3D)@\spxentry{volumen()}\spxextra{método de pr8.4.Objeto3D}}

\begin{fulllineitems}
\phantomsection\label{\detokenize{pr8:pr8.4.Objeto3D.volumen}}
\pysigstartsignatures
\pysiglinewithargsret{\sphinxbfcode{\sphinxupquote{volumen}}}{}{}
\pysigstopsignatures
\sphinxAtStartPar
Calcula el volumen del objeto. Este método debe ser implementado por la subclase.
\begin{quote}\begin{description}
\sphinxlineitem{Muestra}
\sphinxAtStartPar
\sphinxstyleliteralstrong{\sphinxupquote{NotImplementedError}} \textendash{} Si no está implementado en la subclase.

\end{description}\end{quote}

\end{fulllineitems}


\end{fulllineitems}

\index{ObjetoGeometrico (clase en pr8.4)@\spxentry{ObjetoGeometrico}\spxextra{clase en pr8.4}}

\begin{fulllineitems}
\phantomsection\label{\detokenize{pr8:pr8.4.ObjetoGeometrico}}
\pysigstartsignatures
\pysigline{\sphinxbfcode{\sphinxupquote{class\DUrole{w}{ }}}\sphinxcode{\sphinxupquote{pr8.4.}}\sphinxbfcode{\sphinxupquote{ObjetoGeometrico}}}
\pysigstopsignatures
\sphinxAtStartPar
Bases: \sphinxcode{\sphinxupquote{object}}

\sphinxAtStartPar
Clase base para todos los objetos geométricos.
Se utiliza como clase base para los objetos 2D y 3D.

\end{fulllineitems}

\index{Vehiculo (clase en pr8.4)@\spxentry{Vehiculo}\spxextra{clase en pr8.4}}

\begin{fulllineitems}
\phantomsection\label{\detokenize{pr8:pr8.4.Vehiculo}}
\pysigstartsignatures
\pysiglinewithargsret{\sphinxbfcode{\sphinxupquote{class\DUrole{w}{ }}}\sphinxcode{\sphinxupquote{pr8.4.}}\sphinxbfcode{\sphinxupquote{Vehiculo}}}{\sphinxparam{\DUrole{n}{marca}}\sphinxparamcomma \sphinxparam{\DUrole{n}{modelo}}\sphinxparamcomma \sphinxparam{\DUrole{n}{consumo\_combustible}}}{}
\pysigstopsignatures
\sphinxAtStartPar
Bases: \sphinxcode{\sphinxupquote{object}}

\sphinxAtStartPar
Clase base para vehículos. Proporciona métodos para calcular el costo de viaje y obtener las características.
\begin{description}
\sphinxlineitem{Atributos:}
\sphinxAtStartPar
marca (str): Marca del vehículo.
modelo (str): Modelo del vehículo.
consumo\_combustible (float): Consumo de combustible en litros por kilómetro.

\end{description}
\index{caracteristicas() (método de pr8.4.Vehiculo)@\spxentry{caracteristicas()}\spxextra{método de pr8.4.Vehiculo}}

\begin{fulllineitems}
\phantomsection\label{\detokenize{pr8:pr8.4.Vehiculo.caracteristicas}}
\pysigstartsignatures
\pysiglinewithargsret{\sphinxbfcode{\sphinxupquote{caracteristicas}}}{}{}
\pysigstopsignatures
\sphinxAtStartPar
Devuelve las características del vehículo como un string.
\begin{quote}\begin{description}
\sphinxlineitem{Devuelve}
\sphinxAtStartPar
Características del vehículo.

\end{description}\end{quote}

\end{fulllineitems}

\index{costo\_viaje() (método de pr8.4.Vehiculo)@\spxentry{costo\_viaje()}\spxextra{método de pr8.4.Vehiculo}}

\begin{fulllineitems}
\phantomsection\label{\detokenize{pr8:pr8.4.Vehiculo.costo_viaje}}
\pysigstartsignatures
\pysiglinewithargsret{\sphinxbfcode{\sphinxupquote{costo\_viaje}}}{\sphinxparam{\DUrole{n}{distancia}}\sphinxparamcomma \sphinxparam{\DUrole{n}{precio\_combustible}}}{}
\pysigstopsignatures
\sphinxAtStartPar
Calcula el costo de un viaje en función de la distancia y el precio del combustible.
\begin{quote}\begin{description}
\sphinxlineitem{Parámetros}\begin{itemize}
\item {} 
\sphinxAtStartPar
\sphinxstyleliteralstrong{\sphinxupquote{distancia}} \textendash{} Distancia a recorrer en kilómetros.

\item {} 
\sphinxAtStartPar
\sphinxstyleliteralstrong{\sphinxupquote{precio\_combustible}} \textendash{} Precio del combustible por litro.

\end{itemize}

\sphinxlineitem{Devuelve}
\sphinxAtStartPar
El costo del viaje.

\end{description}\end{quote}

\end{fulllineitems}


\end{fulllineitems}



\chapter{Indices y tablas:}
\label{\detokenize{index:indices-y-tablas}}\begin{itemize}
\item {} 
\sphinxAtStartPar
\DUrole{xref,std,std-ref}{genindex}

\item {} 
\sphinxAtStartPar
\DUrole{xref,std,std-ref}{modindex}

\item {} 
\sphinxAtStartPar
\DUrole{xref,std,std-ref}{search}

\end{itemize}


\renewcommand{\indexname}{Índice de Módulos Python}
\begin{sphinxtheindex}
\let\bigletter\sphinxstyleindexlettergroup
\bigletter{p}
\item\relax\sphinxstyleindexentry{pr3.1\_1}\sphinxstyleindexpageref{pr3:\detokenize{module-pr3.1_1}}
\item\relax\sphinxstyleindexentry{pr3.1\_2}\sphinxstyleindexpageref{pr3:\detokenize{module-pr3.1_2}}
\item\relax\sphinxstyleindexentry{pr3.1\_3}\sphinxstyleindexpageref{pr3:\detokenize{module-pr3.1_3}}
\item\relax\sphinxstyleindexentry{pr3.1\_4}\sphinxstyleindexpageref{pr3:\detokenize{module-pr3.1_4}}
\item\relax\sphinxstyleindexentry{pr3.1\_5}\sphinxstyleindexpageref{pr3:\detokenize{module-pr3.1_5}}
\item\relax\sphinxstyleindexentry{pr3.2\_1}\sphinxstyleindexpageref{pr3:\detokenize{module-pr3.2_1}}
\item\relax\sphinxstyleindexentry{pr3.2\_2}\sphinxstyleindexpageref{pr3:\detokenize{module-pr3.2_2}}
\item\relax\sphinxstyleindexentry{pr3.3\_1}\sphinxstyleindexpageref{pr3:\detokenize{module-pr3.3_1}}
\item\relax\sphinxstyleindexentry{pr3.3\_2}\sphinxstyleindexpageref{pr3:\detokenize{module-pr3.3_2}}
\item\relax\sphinxstyleindexentry{pr4.1}\sphinxstyleindexpageref{pr4:\detokenize{module-pr4.1}}
\item\relax\sphinxstyleindexentry{pr4.10}\sphinxstyleindexpageref{pr4:\detokenize{module-pr4.10}}
\item\relax\sphinxstyleindexentry{pr4.2}\sphinxstyleindexpageref{pr4:\detokenize{module-pr4.2}}
\item\relax\sphinxstyleindexentry{pr4.3}\sphinxstyleindexpageref{pr4:\detokenize{module-pr4.3}}
\item\relax\sphinxstyleindexentry{pr4.4}\sphinxstyleindexpageref{pr4:\detokenize{module-pr4.4}}
\item\relax\sphinxstyleindexentry{pr4.5}\sphinxstyleindexpageref{pr4:\detokenize{module-pr4.5}}
\item\relax\sphinxstyleindexentry{pr4.6}\sphinxstyleindexpageref{pr4:\detokenize{module-pr4.6}}
\item\relax\sphinxstyleindexentry{pr4.7}\sphinxstyleindexpageref{pr4:\detokenize{module-pr4.7}}
\item\relax\sphinxstyleindexentry{pr4.8}\sphinxstyleindexpageref{pr4:\detokenize{module-pr4.8}}
\item\relax\sphinxstyleindexentry{pr4.9}\sphinxstyleindexpageref{pr4:\detokenize{module-pr4.9}}
\item\relax\sphinxstyleindexentry{pr5.1}\sphinxstyleindexpageref{pr5:\detokenize{module-pr5.1}}
\item\relax\sphinxstyleindexentry{pr5.10}\sphinxstyleindexpageref{pr5:\detokenize{module-pr5.10}}
\item\relax\sphinxstyleindexentry{pr5.11}\sphinxstyleindexpageref{pr5:\detokenize{module-pr5.11}}
\item\relax\sphinxstyleindexentry{pr5.2}\sphinxstyleindexpageref{pr5:\detokenize{module-pr5.2}}
\item\relax\sphinxstyleindexentry{pr5.3}\sphinxstyleindexpageref{pr5:\detokenize{module-pr5.3}}
\item\relax\sphinxstyleindexentry{pr5.4}\sphinxstyleindexpageref{pr5:\detokenize{module-pr5.4}}
\item\relax\sphinxstyleindexentry{pr5.5}\sphinxstyleindexpageref{pr5:\detokenize{module-pr5.5}}
\item\relax\sphinxstyleindexentry{pr5.6}\sphinxstyleindexpageref{pr5:\detokenize{module-pr5.6}}
\item\relax\sphinxstyleindexentry{pr5.7}\sphinxstyleindexpageref{pr5:\detokenize{module-pr5.7}}
\item\relax\sphinxstyleindexentry{pr5.8}\sphinxstyleindexpageref{pr5:\detokenize{module-pr5.8}}
\item\relax\sphinxstyleindexentry{pr5.9}\sphinxstyleindexpageref{pr5:\detokenize{module-pr5.9}}
\item\relax\sphinxstyleindexentry{pr6.1\_1}\sphinxstyleindexpageref{pr6:\detokenize{module-pr6.1_1}}
\item\relax\sphinxstyleindexentry{pr6.1\_2}\sphinxstyleindexpageref{pr6:\detokenize{module-pr6.1_2}}
\item\relax\sphinxstyleindexentry{pr6.1\_3}\sphinxstyleindexpageref{pr6:\detokenize{module-pr6.1_3}}
\item\relax\sphinxstyleindexentry{pr6.1\_4}\sphinxstyleindexpageref{pr6:\detokenize{module-pr6.1_4}}
\item\relax\sphinxstyleindexentry{pr6.1\_5}\sphinxstyleindexpageref{pr6:\detokenize{module-pr6.1_5}}
\item\relax\sphinxstyleindexentry{pr6.1\_6}\sphinxstyleindexpageref{pr6:\detokenize{module-pr6.1_6}}
\item\relax\sphinxstyleindexentry{pr6.1\_7}\sphinxstyleindexpageref{pr6:\detokenize{module-pr6.1_7}}
\item\relax\sphinxstyleindexentry{pr6.2\_1}\sphinxstyleindexpageref{pr6:\detokenize{module-pr6.2_1}}
\item\relax\sphinxstyleindexentry{pr6.2\_2}\sphinxstyleindexpageref{pr6:\detokenize{module-pr6.2_2}}
\item\relax\sphinxstyleindexentry{pr6.2\_3}\sphinxstyleindexpageref{pr6:\detokenize{module-pr6.2_3}}
\item\relax\sphinxstyleindexentry{pr6.2\_4}\sphinxstyleindexpageref{pr6:\detokenize{module-pr6.2_4}}
\item\relax\sphinxstyleindexentry{pr6.3\_1}\sphinxstyleindexpageref{pr6:\detokenize{module-pr6.3_1}}
\item\relax\sphinxstyleindexentry{pr7.1}\sphinxstyleindexpageref{pr7:\detokenize{module-pr7.1}}
\item\relax\sphinxstyleindexentry{pr7.2}\sphinxstyleindexpageref{pr7:\detokenize{module-pr7.2}}
\item\relax\sphinxstyleindexentry{pr7.3}\sphinxstyleindexpageref{pr7:\detokenize{module-pr7.3}}
\item\relax\sphinxstyleindexentry{pr8.1}\sphinxstyleindexpageref{pr8:\detokenize{module-pr8.1}}
\item\relax\sphinxstyleindexentry{pr8.2}\sphinxstyleindexpageref{pr8:\detokenize{module-pr8.2}}
\item\relax\sphinxstyleindexentry{pr8.3}\sphinxstyleindexpageref{pr8:\detokenize{module-pr8.3}}
\item\relax\sphinxstyleindexentry{pr8.4}\sphinxstyleindexpageref{pr8:\detokenize{module-pr8.4}}
\end{sphinxtheindex}

\renewcommand{\indexname}{Índice}
\printindex
\end{document}